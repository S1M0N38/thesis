\begin{frame}{Constraints classification}
  \begin{definition} \vspace{0.3em}
    In Hamiltonian mechanics, \emph{constraints} are relation between
    coordinates and conjugate momenta.
  \end{definition}
  \begin{alertblock}{First classification}
    \begin{itemize}
      \item \emph{First class constraints}: the ``physical'' ones (e.g.\ train
        on a railway, particle on a plane).
      \item \emph{Second class constraints}: related to gauge freedom.
    \end{itemize}
  \end{alertblock}
  \begin{remark} \vspace{0.3em}
    The Poisson Bracket of a first class constraint with all the
    other constraints vanishes on the constraints surface in the phase space
    (i.e.\ the surface where all constraints are satisfied).
  \end{remark}
\end{frame}

\begin{frame}{Constraints classification}
  \begin{alertblock}{Second classification}
    \begin{itemize}
      \item \emph{Primary constraints}: holds independently from the equations
        of motion.
      \begin{equation*} \label{eq:constraint}
        \phi_1(Q, P) = 0
      \end{equation*}
      \item \emph{Secondary constraints}: derived from the primary one
        imposing the preservation of the constraints during the evolution of
        the system.
    \end{itemize}
    This generate a series of constraints called \emph{consistency
      relations}~\footnote{
      The weak equality symbol ``$\approx$'' highlight the fact that these
      relations vanish only on the hypersurface where all constraints are
      satisfied.
    }
    \begin{equation*} \label{eq:consistency_relations}
      \phi_1 = 0 \quad \Rightarrow \quad
      \left\{ \phi_1 , H \right\} =: \phi_2 \approx 0 \quad \Rightarrow \quad
      \left\{ \phi_2 , H \right\} =: \phi_3 \approx 0 \quad \Rightarrow \quad
      \ldots
    \end{equation*}
  \end{alertblock}
\end{frame}

\begin{frame}{Constrained Hamiltonians}
  \begin{alertblock}{Introducing m-constraints in n-th order Hamiltonian}
  \end{alertblock}
  Constraints can be imposed in the Hamiltonian with the usage of
  auxiliary variables $\lambda_i$ in the corresponding Lagrangian
  \begin{equation*}
    L = L(q, \dot{q}, \ddot{q}, \ldots, q^{(n)},
      \lambda_1, \lambda_2, \ldots, \lambda_m)
    \quad \text{where} \quad
    \frac{\partial L}{\partial \lambda_i} = 0 \quad i=1, 2, \ldots, m
  \end{equation*}

  Canonical coordinates has to be chosen also for $\lambda_i$
  \begin{equation*} \label{eq:def_canonical_coordinates_lambda}
    \Lambda_{i}:= \lambda_{i}
    \quad \leftrightarrow \quad
    \Pi_{i} := \frac{\delta L}{\delta \dot{\lambda_{i}}} = 0
    \qquad i = 1, 2, \ldots, m
  \end{equation*}
  where
  \begin{equation*}
    \phi_{1, i}: \Pi_i = 0
  \end{equation*}
  are indeed the primary constraints.
\end{frame}

\begin{frame}{Constrained Hamiltonians}
  To the unconstrained $\tilde{H}$ are now added the auxiliary
  variables terms becoming
  \begin{equation*}
    \tilde{H} :=
    \sum_{j=1}^{n} \frac{\delta L}{\delta q^{(j)}} q^{(j)} +
    \sum_{i=1}^{m} \frac{\delta L}{\delta \dot{\lambda_i}} \dot{\lambda_i} -
    L(q, \ldots, q^{(n)}, \lambda_1, \ldots, \lambda_m)
  \end{equation*}

  Depending on the explicit form of the constraints in the Lagrangian,
  different number of secondary constraints can be found using the consistency
  relations.

  Finally, constraints relation can be organize as follows
  \begin{equation*}
    \begin{cases}
      \Lambda_i =\ f_i(Q_1, \ldots, Q_n, P_n) \\
      \Pi_i =\ 0
    \end{cases}
    \qquad i = 1, 2, \ldots, m
  \end{equation*}
  and substitute in $\tilde{H}$ obtaining the \emph{constrained
  Hamiltonian}
  \begin{equation*}
    H =\ P_n h + P_{n-1} Q_n + \cdots + P_1 Q_2
       - L ( Q_1, Q_2, \ldots, h, f_1, \ldots, f_m)
  \end{equation*}
\end{frame}
