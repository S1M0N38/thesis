\begin{frame}{First order vs Second order Lagrangian}
  We would like to find the differences between systems that can be
  described by these two Lagrangians:
  \vspace{0.5em}
  \begin{columns}
    \begin{column}{0.4\textwidth}
      \begin{center}
        \alert{First order system}
        \begin{equation*}
          L_1(q, \dot{q}) = \frac{\dot{q}^2}{2} - \alpha(q)
        \end{equation*}
      \end{center}
    \end{column}
    \begin{column}{0.4\textwidth}
      \begin{center}
        \alert{Second order system}
        \begin{equation*}
          L_2(q, \ddot{q}) = \frac{\ddot{q}^2}{2} - \alpha(q)
        \end{equation*}
      \end{center}
    \end{column}
  \end{columns}
\end{frame}

\begin{frame}{First order Lagrangian}
  \begin{equation*}
    L_1(q, \dot{q}) = \frac{\dot{q}^2}{2} - \alpha(q)
  \end{equation*}
  \vspace{0.2em}
  \begin{equation*} \label{eq: first-order_motion_eq_lagr}
    \frac{\partial L_1}{\partial q} -
    \frac{d}{dt}\frac{\partial L_1}{\partial \dot{q}} +
    \underbrace{
    \frac{d^2}{dt^2}\frac{\partial L_1}{\partial \ddot{q}} - \ldots }_0 = 0
    \qquad \Rightarrow \qquad
    \ddot{q} = - \frac{d\alpha(q)}{dq}
  \end{equation*}
  \begin{equation*}
    \begin{cases}
      Q_1 := q \\
      P_1 := \frac{\delta L_1}{\delta \dot{q}} = \dot{q}
    \end{cases}
    \qquad
    \tilde{H}_1(q, \dot{q}) :=
    \frac{\delta L_1}{\delta \dot{q}} \dot{q} - L_1(q, \dot{q})
  \end{equation*}
  \metroset{block=fill}
  \begin{block}{First order Hamiltonian}
    \begin{equation*} \label{eq: first-order_motion_eq_ham}
      H_1(Q_1, P_1) = \frac{P_1^2}{2} + \alpha(Q_1) \qquad
    \end{equation*}
  \end{block}
\end{frame}

\begin{frame}{Second order Lagrangian}
  \begin{equation*}
    L_2(q, \ddot{q}) = \frac{\ddot{q}^2}{2} - \alpha(q)
  \end{equation*}
  \vspace{0.2em}
  \begin{equation*} \label{eq: second-order_motion_eq_lagr}
    \frac{\partial L_2}{\partial q} -
    \frac{d}{dt}\frac{\partial L_2}{\partial \dot{q}} +
    \frac{d^2}{dt^2}\frac{\partial L_2}{\partial \ddot{q}} -
    \underbrace{
      \frac{d^3}{dt^3}\frac{\partial L_2}{\partial q^{(3)}} + \ldots
    }_0 = 0
    \quad \Rightarrow \quad
    q^{(4)} = - \frac{d\alpha(q)}{dq}
  \end{equation*}
  \begin{equation*}
    \begin{cases}
      Q_1 := q \\
      P_1 := \frac{\delta L_2}{\delta \dot{q}}
           = \frac{\partial L_2}{\partial \dot{q}} -
             \frac{d}{dt} \left( \frac{\partial L_2}{\partial \ddot{q}} \right)
           = 0 - \frac{d}{dt} \left( \ddot{q} \right) = -q^{(3)}
    \end{cases}
    \begin{cases}
      Q_2 := \dot{q} \\
      P_2 := \frac{\delta L_2}{\delta \ddot{q}}
           = \frac{\partial L_2}{\partial \ddot{q}} = \ddot{q}
    \end{cases}
  \end{equation*}

  Therefore
  \begin{equation*}
    \begin{cases}
      q        = Q_1 \\
      \dot{q}  = Q_2 \\
      \ddot{q} = P_2
    \end{cases}
    \qquad
    \begin{cases}
      \frac{\delta L_2}{\delta \dot{q}} =
        \frac{\partial L_2}{\partial \dot{q}} -
        \frac{d}{dt} \left( \frac{\partial L_2}{\partial \ddot{q}} \right) =
        P_1 \\
      \frac{\delta L_2}{\delta \ddot{q}} =
        \frac{\partial L_2}{\partial \ddot{q}} =
        P_2
    \end{cases}
  \end{equation*}
\end{frame}

\begin{frame}{Second order Lagrangian}
  \begin{equation}\label{eq:canonical_variable_second_order}
    \begin{cases}
      q        = Q_1 \\
      \dot{q}  = Q_2 \\
      \ddot{q} = P_2
    \end{cases}
    \qquad
    \begin{cases}
      \frac{\delta L_2}{\delta \dot{q}} =
        \frac{\partial L_2}{\partial \dot{q}} -
        \frac{d}{dt} \left( \frac{\partial L_2}{\partial \ddot{q}} \right) =
        P_1 \\
      \frac{\delta L_2}{\delta \ddot{q}} =
        \frac{\partial L_2}{\partial \ddot{q}} =
        P_2
    \end{cases}
  \end{equation}
  \begin{equation}\label{eq:h_tilda_second_order}
    \tilde{H}_2(q, \dot{q}) :=
      \frac{\delta L_2}{\delta \dot{q}} \dot{q} +
      \frac{\delta L_2}{\delta \ddot{q}} \ddot{q} -
      L_2(q, \ddot{q})
      = \frac{\delta L_2}{\delta \dot{q}} \dot{q} +
      \frac{\delta L_2}{\delta \ddot{q}} \ddot{q} -
      \frac{\ddot{q}^2}{2} + \alpha(q)
  \end{equation}

  Substituting~\eqref{eq:canonical_variable_second_order}
  in~\eqref{eq:h_tilda_second_order} we get the Hamiltonian
  \begin{align*} \label{eq: second-order_motion_eq_ham}
    H_2(Q_1, Q_2, P_1, P_2)
      &= P_1 Q_2 + P_2 P_2 - \frac{P_2^2}{2} + \alpha(Q_1) \\
      &= P_1 Q_2 + \frac{P_2^2}{2} + \alpha(Q_1)
  \end{align*}

  \metroset{block=fill}
  \begin{block}{Second order Hamiltonian}
    \begin{equation*} \label{eq: second-order_motion_eq_ham}
      H_2(Q_1, Q_2, P_1, P_2) = P_1Q_2 + \frac{P_2^2}{2} + \alpha(Q_1)
    \end{equation*}
  \end{block}
\end{frame}

\begin{frame}{First order vs Second order Hamiltonian}
  \vspace{0.5em}
  \begin{columns}
    \begin{column}{0.5\textwidth}
      \begin{center}
        \alert{First order system} \vspace{0.1em}
        \begin{equation*}
          L_1(q, \dot{q}) = \frac{\dot{q}^2}{2} - \alpha(q)
        \end{equation*}
        \begin{equation*} \label{eq: first-order_motion_eq_ham}
          H_1(Q_1, P_1) = \frac{P_1^2}{2} + \alpha(Q_1) \qquad
        \end{equation*}
      \end{center}
    \end{column}
    \begin{column}{0.5\textwidth}
      \begin{center}
        \alert{Second order system} \vspace{0.1em}
        \begin{equation*}
          L_2(q, \ddot{q}) = \frac{\ddot{q}^2}{2} - \alpha(q)
        \end{equation*}
        \begin{equation*} \label{eq: second-order_motion_eq_ham}
          H_2(Q_1, Q_2, P_1, P_2) = P_1Q_2 + \frac{P_2^2}{2} + \alpha(Q_1)
        \end{equation*}
      \end{center}
    \end{column}
  \end{columns}
  \vspace{2.0em}
  There is a significant difference between the two spectra of $H_1$ and
  $H_2$: the first is bounded from below while the latter is not (and both are
  not bounded from above).
\end{frame}
