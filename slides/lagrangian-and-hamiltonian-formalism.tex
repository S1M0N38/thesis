\begin{frame}{Functional derivative}
  Define the \emph{action functional} as
  \begin{equation*}
    S[q] := \int_{t_1}^{t_2} dt \,
    L \left(q, \dot{q}, \ddot{q}, \ldots, q^{(n)} \right)
    \qquad \text{where} \qquad q^{(k)} := \frac{d^k q(t)}{dt^k}
  \end{equation*}

  Produce the \emph{k-th order perturbation on $q^{(k)}$}
  \begin{equation*}
    q^{(k)} \rightarrow \frac{d^k q(t)}{dt^k} + \frac{d^k \delta q(t)}{dt^k} =
    q^{(k)} + \delta q^{(k)}
  \end{equation*}

  and define the \emph{variation of the action functional} as
  \begin{equation*}
    \delta S[q] := S[q + \delta q] - S[q]
  \end{equation*}

  Considering $\delta q$ infinitesimal and keeping the ends points fixed we
  get % \\ (i.e. $\delta y^{k}(t_1) = \delta y^{k}(t_2) = 0$ for all k)
  \begin{equation*}
    \delta S[q] = \int_{t_1}^{t_2} dt \,
    \sum_{k=0}^{n} {(-1)}^k \frac{d^k}{dt^k}
    \left(\frac{\partial L}{\partial q^{(k)}}\right) \delta q
  \end{equation*}
\end{frame}

\begin{frame}{Functional derivative}
  Noting the similarities with the differential in multi-variable calculus
  \begin{equation*}
    df(x) = \sum_{k=1}^{n}
    \left\{\frac{\partial f(x)}{\partial x_k} dx_k\right\}
    \quad \longleftrightarrow \quad
    \delta S[q] = \int_{t_1}^{t_2} dt \,
    \left\{ \frac{\delta S[q]}{\delta q} \delta q \right\}
  \end{equation*}

  the \emph{functional derivative with respect to $q$} can be define
  \begin{equation*}
    \frac{\delta S[q]}{\delta q}:=
    \sum_{k=0}^{n} {(-1)}^k \frac{d^k}{dt^k}
    \left(\frac{\partial L}{\partial q^{(k)}}\right)
  \end{equation*}

  Generalization and abuse of notation lead to writing
  \metroset{block=fill}
  \begin{block}{Functional derivative with respect to $q^{(i)}$}
    \begin{equation*}
      \frac{\delta L}{\delta q^{(i)}}:=
      \sum_{k=0}^{n} {(-1)}^k \frac{d^k}{dt^k}
      \left(\frac{\partial L}{\partial q^{(i+k)}}\right)
    \end{equation*}
  \end{block}
\end{frame}

\begin{frame}{Lagrangian formalism}
  \begin{alertblock}{Hamilton's principle}
    \vspace{0.5em}
    The motion of the system is a stationary point of the action
    functional~\cite{Goldstein11_Ham_principle}
    \begin{equation*}
      \frac{\delta S[q]}{\delta q} = 0
    \end{equation*}
  \end{alertblock}
  \begin{alertblock}{Euler-Lagrange equation}
    \vspace{0.5em}
    \begin{equation*}
      \Rightarrow \qquad
      \frac{\delta S[q]}{\delta q} =
      \frac{\partial L}{\partial q} -
      \frac{d}{dt}\frac{\partial L}{\partial \dot{q}} +
      \frac{d^2}{dt^2}\frac{\partial L}{\partial \ddot{q}} -
      \ldots = 0
    \end{equation*}

    Notice that in the Euler-Lagrange equation for a n-th order Lagrangian, the
    generalized coordinates ($q$) appear to be derived up to the 2n-th order
    $\left(\frac{d^n}{dt^n} \frac{\partial L}{\partial q^{(n)}}\right)$.
  \end{alertblock}
\end{frame}

\begin{enumerate}
\begin{frame}{Hamiltonian formalism}
  \begin{alertblock}{From Lagrange to Hamilton in 3 steps}
    \vspace{0.5em}
    \item Define i-th \emph{canonical coordinates as} $(i = 1, \ldots, n)$
      \begin{equation*} \label{eq:def_canonical_coordinates}
        \begin{cases}
          Q_{(i)} := q^{(i-1)} = \frac{d^{\, i-1} q}{dt^{\, i-1}} \\
          P_{(i)} := \frac{\delta L}{\delta q^{(i)}}
            = \frac{\partial L}{\partial q^{(i)}}
            - \frac{d}{dt} \frac{\partial L}{\partial q^{(i + 1)}}
            + \frac{d^2}{dt^2} \frac{\partial L}{\partial q^{(i + 2)}} - \dots
        \end{cases}
        \text{with } i = 1, 2, \ldots, n
      \end{equation*}
    \item Define $\tilde{H}$ as
      \begin{equation*} \label{eq:Ham_in_q}
        \tilde{H}(q, \dot{q}, \ldots, q^{(n)}) :=
        \sum_{i=1}^{n} \frac{\delta L}{\delta q^{(i)}} q^{(i)} -
        L(q, \dot{q}, \ldots, q^{(n)})
      \end{equation*}
      Under the hypothesis of regular Lagrangian, relation for $q^{(n)}$ can
      be inverted
      \begin{equation*} \label{eq:qn=h}
        \frac{\delta L}{\delta q^{(n)}} = P_n
        \quad \Rightarrow \quad
        q^{(n)} = h(Q_{(1)}, \ldots, Q_{(n)}, P_{(n)}) \equiv h
      \end{equation*}
  \end{alertblock}
\end{frame}

\begin{frame}{Hamiltonian formalism}
  \begin{alertblock}{}
      Substituting in the expression for $\tilde{H}$ we get the
      \emph{Hamiltonian}
      \begin{equation*} \label{eq:general_hamiltonian}
        H = P_n h + P_{n-1} Q_n + \cdots +
            P_1 Q_2 - L(Q_1, Q_2, \ldots, h)
      \end{equation*}
    \item The Hamiltonian generates the time evolution of any function of
      canonical variables $F(Q_i, P_i)$ via the \emph{Poisson
      Brackets}~\cite{Chen13}
      \begin{equation*} \label{eq:poisson_bracket_evolution}
        \dot{F}(Q_i, P_i) = \left\{ F(Q_i, P_i), H \right\}
      \end{equation*}
      Poisson Brackets, in canonical coordinates, are defined as
      \begin{equation*} \label{eq:possion_braket}
        \left\{ f, g \right\} := \sum_{i=1}^{n} \left(
          \frac{\partial f}{\partial Q_{(i)}}
          \frac{\partial g}{\partial P_{(i)}} -
          \frac{\partial f}{\partial P_{(i)}}
          \frac{\partial g}{\partial Q_{(i)}}
        \right)
      \end{equation*}
  \end{alertblock}
\end{frame}
\end{enumerate}
