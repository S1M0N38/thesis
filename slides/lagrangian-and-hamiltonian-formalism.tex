\begin{frame}{Functional derivative}
  Define the \emph{action functional} as
  \begin{equation*}
    S[q] := \int_{t_1}^{t_2} dt \,
    L \left(q, \dot{q}, \ddot{q}, \ldots, q^{(n)} \right)
    \qquad \text{where} \qquad q^{(k)} := \frac{d^k q(t)}{dt^k}
  \end{equation*}

  Produce the \emph{k-th order perturbation on $q^{(k)}$}
  \begin{equation*}
    q^{(k)} \rightarrow \frac{d^k q(t)}{dt^k} + \frac{d^k \delta q(t)}{dt^k} =
    q^{(k)} + \delta q^{(k)}
  \end{equation*}

  and define the \emph{variation of the action functional} as
  \begin{equation*}
    \delta S[q] := S[q + \delta q] - S[q]
  \end{equation*}

  Considering $\delta q$ infinitesimal and keeping the ends points fixed we
  get % \\ (i.e. $\delta y^{k}(t_1) = \delta y^{k}(t_2) = 0$ for all k)
  \begin{equation*}
    \delta S[q] = \int_{t_1}^{t_2} dt \,
    \sum_{k=0}^{n} {(-1)}^k \frac{d^k}{dt^k}
    \left(\frac{\partial L}{\partial q^{(k)}}\right) \delta q
  \end{equation*}
\end{frame}

\begin{frame}{Functional derivative}
  Noting the similarities with the differential in multi-variable calculus
  \begin{equation*}
    df(x) = \sum_{k=1}^{n}
    \left\{\frac{\partial f(x)}{\partial x_k} dx_k\right\}
    \quad \longleftrightarrow \quad
    \delta S[q] = \int_{t_1}^{t_2} dt \,
    \left\{ \frac{\delta S[q]}{\delta q} \delta q \right\}
  \end{equation*}

  the \emph{functional derivative with respect to $q$} can be define
  \begin{equation*}
    \frac{\delta S[q]}{\delta q}:=
    \sum_{k=0}^{n} {(-1)}^k \frac{d^k}{dt^k}
    \left(\frac{\partial L}{\partial q^{(k)}}\right)
  \end{equation*}

  Generalization and abuse of notation lead to writing
  \metroset{block=fill}
  \begin{block}{Functional derivative with respect to $q^{(i)}$}
    \begin{equation*}
      \frac{\delta L}{\delta q^{(i)}}:=
      \sum_{k=0}^{n} {(-1)}^k \frac{d^k}{dt^k}
      \left(\frac{\partial L}{\partial q^{(i+k)}}\right)
    \end{equation*}
  \end{block}
\end{frame}

\begin{frame}{Lagrangian formalism}
  \begin{alertblock}{Hamilton's principle}
    \vspace{0.5em}
    The motion of the system is a stationary point of the action
    functional~\cite{Goldstein11_Ham_principle}
    \begin{equation*}
      \frac{\delta S[q]}{\delta q} = 0
    \end{equation*}
  \end{alertblock}
  \begin{alertblock}{Euler-Lagrange equation}
    \vspace{0.5em}
    \begin{equation*}
      \Rightarrow \qquad
      \frac{\delta S[q]}{\delta q} =
      \frac{\partial L}{\partial q} -
      \frac{d}{dt}\frac{\partial L}{\partial \dot{q}} +
      \frac{d^2}{dt^2}\frac{\partial L}{\partial \ddot{q}} -
      \ldots = 0
    \end{equation*}
  \end{alertblock}
\end{frame}

\begin{enumerate}
\begin{frame}{Hamiltonian formalism}
  \begin{alertblock}{From Lagrange to Hamilton in 3 step}
    \vspace{0.5em}
    \item Define i-th \emph{canonical coordinates as}
      \begin{equation*} \label{eq:def_canonical_coordinates}
        Q_{(i)} := q^{(i-1)} = \frac{d^{\, i-1} q}{dt^{\, i-1}}
        \quad \leftrightarrow \quad
        P_{(i)} := \frac{\delta L}{\delta q^{(i)}}
        \qquad i = 1, \ldots, n
      \end{equation*}
    \item Define $\tilde{H}$ as
      \begin{equation*} \label{eq:Ham_in_q}
        \tilde{H}(q, \dot{q}, \ldots, q^{(n)}) :=
        \sum_{i=1}^{n} \frac{\delta L}{\delta q^{(i)}} q^{(i)} -
        L(q, \dot{q}, \ldots, q^{(n)})
      \end{equation*}
      Under the hypothesis of regular Lagrangian, relation for $q^{(n)}$ can
      be inverted
      \begin{equation*} \label{eq:qn=h}
        \frac{\delta L}{\delta q^{(n)}} = P_n
        \quad \Rightarrow \quad
        q^{(n)} = h(Q_{(1)}, \ldots, Q_{(n)}, P_{(n)}) \equiv h
      \end{equation*}
  \end{alertblock}
\end{frame}

\begin{frame}{Hamiltonian formalism}
  \begin{alertblock}{}
      Substituting in the expression for $\tilde{H}$ we get the
      \emph{Hamiltonian}
      \begin{equation*} \label{eq:general_hamiltonian}
        H = P_n h + P_{n-1} Q_n + \cdots +
            P_1 Q_2 - L(Q_1, Q_2, \ldots, h)
      \end{equation*}
    \item The Hamiltonian generates the time evolution of any function of
      canonical variables $F(Q_i, P_i)$ via the \emph{Poisson
      Brackets}~\cite{Chen13}
      \begin{equation*} \label{eq:poisson_bracket_evolution}
        \dot{F}(Q_i, P_i) = \left\{ F(Q_i, P_i), H \right\}
      \end{equation*}
      Poisson Brackets, in canonical coordinates, are defined as
      \begin{equation*} \label{eq:possion_braket}
        \left\{ f, g \right\} := \sum_{i=1}^{n} \left(
          \frac{\partial f}{\partial Q_{(i)}}
          \frac{\partial g}{\partial P_{(i)}} -
          \frac{\partial f}{\partial P_{(i)}}
          \frac{\partial g}{\partial Q_{(i)}}
        \right)
      \end{equation*}
  \end{alertblock}
\end{frame}
\end{enumerate}

\begin{frame}{Motivation behind the definition of canonical variables}
  \begin{alertblock}{Motivation behind the definition of canonical variables}
    \vspace{0.5em}
    Consider a third-order one-dimensional Lagrangian as example
    \begin{equation*}
      L\left( q, \dot{q}, \ddot{q}, q^{(3)} \right)
    \end{equation*}
    and the corresponding Euler-Lagrange equation
    \begin{equation*} \label{eq:third_order_euler_lagrangian}
      \frac{\partial L}{\partial q} -
      \frac{d}{dt}\frac{\partial L}{\partial \dot{q}} +
      \frac{d^2}{dt^2}\frac{\partial L}{\partial \ddot{q}} -
      \frac{d^3}{dt^3}\frac{\partial L}{\partial q^{(3)}} +
      \underbrace{
      \frac{d^4}{dt^4}\frac{\partial L}{\partial q^{(4)}} - \ldots }_0 = 0
    \end{equation*}

    Collecting time derivatives, Euler-Lagrange equation can be written as
    \begin{equation*} \label{eq:third_order_euler_lagrangian_with_parts}
      \frac{\partial L}{\partial q} -
      \frac{d}{dt} \left(
      \frac{\partial L}{\partial \dot{q}} -
      \frac{d}{dt} \left(
      \frac{\partial L}{\partial \ddot{q}} -
      \frac{d}{dt} \left(
      \frac{\partial L}{\partial q^{(3)}}
      \right)\right)\right) = 0
    \end{equation*}
    in which generalized coordinates appear up to sixth-derivation order.
  \end{alertblock}
\end{frame}

\begin{frame}{Motivation behind the definition of canonical variables}
  \begin{alertblock}{}
    In order to translating Lagrangian formalism in the Hamiltonian one, we
    need a \emph{map}~\footnote{
      There is no need to map $q^{(6)}$ because it appears in Euler-Lagrange
      equation as a simple derivation of terms depending on $q^{(5)}$. When
      the expression for $q^{(5)}$ in terms of canonical variables is known,
      $q^{(6)}$ is also known.
    }:
    \begin{equation*}
      q^{(i)} \quad i = 0, 1, 2, 3, 4, 5
      \quad \longrightarrow \quad
      (P_i , Q_i) \quad i = 1, 2, 3
    \end{equation*}
    We choose to map $q$, $\dot{q}$, $\ddot{q}$ into $Q_1$, $Q_2$, $Q_3$
    \begin{equation*}\label{eq:def_Q}
      \begin{cases}
        Q_1 := q \\
        Q_2 := \dot{q} \\
        Q_3 := \ddot{q} \\
      \end{cases}
    \end{equation*}
  \end{alertblock}
\end{frame}

\begin{frame}{Motivation behind the definition of canonical variables}
  \begin{alertblock}{}
    To map $q^{(3)}$, $q^{(4)}$, $q^{(5)}$ we observe that
    \begin{equation*} \label{eq:third_order_euler_lagrangian_h_with_parts}
      \frac{\partial L}{\partial q} -
      \frac{d}{dt} \biggl(
      \underbrace{
      \frac{\partial L}{\partial \dot{q}} -
      \frac{d}{dt} \biggl(
      \underbrace{
      \frac{\partial L}{\partial \ddot{q}} -
      \frac{d}{dt} \biggl(
      \overbrace{
      \frac{\partial L}{\partial q^{(3)}}
      }^{\text{depends on } q^{(3)}} \biggr)
      }_{\text{depends on } q^{(4)}} \biggr)
      }_{\text{depends on } q^{(5)}} \biggr)
      = 0
    \end{equation*}

    So it is convenient to define
    \begin{equation*}\label{eq:def_P}
      \begin{cases}
        P_3 := \text{part that depends on } q^{(3)}
             = \frac{\partial L}{\partial q^{(3)}} \\
        P_2 := \text{part that depends on } q^{(4)}
             = \frac{\partial L}{\partial \ddot{q}} -
               \frac{d}{dt} \left(
                 \frac{\partial L}{\partial q^{(3)}}
               \right) \\
        P_1 := \text{part that depends on } q^{(5)}
             = \frac{\partial L}{\partial \dot{q}} -
               \frac{d}{dt} \left(
                 \frac{\partial L}{\partial \ddot{q}} -
                 \frac{d}{dt} \left(
                   \frac{\partial L}{\partial q^{(3)}}
                  \right)
                \right) \\
      \end{cases}
    \end{equation*}
  \end{alertblock}
\end{frame}

\begin{frame}{Motivation behind the definition of canonical variables}
  \begin{alertblock}{}
    Definition for $P_i$ can be rewritten in a more compact notation employing
    the \emph{functional derivative}
    \begin{equation*}
      \begin{cases}
        P_3  = \frac{\partial L}{\partial q^{(3)}}
             = \frac{\delta L}{\delta q^{(3)}} \\
        P_2  = \frac{\partial L}{\partial \ddot{q}} -
               \frac{d}{dt} \left(
                 \frac{\partial L}{\partial q^{(3)}}
               \right)
             = \frac{\delta L}{\delta \ddot{q}} \\
        P_1  = \frac{\partial L}{\partial \dot{q}} -
               \frac{d}{dt} \left(
                 \frac{\partial L}{\partial \ddot{q}} -
                 \frac{d}{dt} \left(
                   \frac{\partial L}{\partial q^{(3)}}
                  \right)
                \right)
             = \frac{\delta L}{\delta \dot{q}} \\
      \end{cases}
    \end{equation*}

    Generalizing to the nth-order Lagrangian we obtain the previous definition
    of canonical variables
    \begin{equation*}
      Q_{(i)} := q^{(i-1)} = \frac{d^{\, i-1} q}{dt^{\, i-1}}
      \quad \leftrightarrow \quad
      P_{(i)} := \frac{\delta L}{\delta q^{(i)}}
      \qquad i = 1, \ldots, n
    \end{equation*}
  \end{alertblock}
\end{frame}
