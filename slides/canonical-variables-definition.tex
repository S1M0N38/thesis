\begin{frame}{Motivation behind the definition of canonical variables}
  \begin{alertblock}{Motivation behind the definition of canonical variables}
    \vspace{0.5em}
    Consider a third-order one-dimensional Lagrangian as example
    \begin{equation*}
      L\left( q, \dot{q}, \ddot{q}, q^{(3)} \right)
    \end{equation*}
    and the corresponding Euler-Lagrange equation
    \begin{equation*} \label{eq:third_order_euler_lagrangian}
      \frac{\partial L}{\partial q} -
      \frac{d}{dt}\frac{\partial L}{\partial \dot{q}} +
      \frac{d^2}{dt^2}\frac{\partial L}{\partial \ddot{q}} -
      \frac{d^3}{dt^3}\frac{\partial L}{\partial q^{(3)}} +
      \underbrace{
      \frac{d^4}{dt^4}\frac{\partial L}{\partial q^{(4)}} - \ldots }_0 = 0
    \end{equation*}

    Collecting time derivatives, Euler-Lagrange equation can be written as
    \begin{equation*} \label{eq:third_order_euler_lagrangian_with_parts}
      \frac{\partial L}{\partial q} -
      \frac{d}{dt} \left(
      \frac{\partial L}{\partial \dot{q}} -
      \frac{d}{dt} \left(
      \frac{\partial L}{\partial \ddot{q}} -
      \frac{d}{dt} \left(
      \frac{\partial L}{\partial q^{(3)}}
      \right)\right)\right) = 0
    \end{equation*}
    in which generalized coordinates appear up to sixth-derivation order.
  \end{alertblock}
\end{frame}

\begin{frame}{Motivation behind the definition of canonical variables}
  \begin{alertblock}{}
    In order to translate Lagrangian formalism into the Hamiltonian one, we
    need a \emph{map}~\footnote{
      There is no need to map $q^{(6)}$ because it appears in Euler-Lagrange
      equation as a simple derivation of terms depending on $q^{(5)}$. When
      the expression for $q^{(5)}$ in terms of canonical variables is known,
      $q^{(6)}$ is also known.
    }:
    \begin{equation*}
      q^{(i)} \quad i = 0, 1, 2, 3, 4, 5
      \quad \longrightarrow \quad
      (P_i , Q_i) \quad i = 1, 2, 3
    \end{equation*}
    We choose to map $q$, $\dot{q}$, $\ddot{q}$ into $Q_1$, $Q_2$, $Q_3$
    \begin{equation*}\label{eq:def_Q}
      \begin{cases}
        Q_1 := q \\
        Q_2 := \dot{q} \\
        Q_3 := \ddot{q} \\
      \end{cases}
    \end{equation*}
  \end{alertblock}
\end{frame}

\begin{frame}{Motivation behind the definition of canonical variables}
  \begin{alertblock}{}
    To map $q^{(3)}$, $q^{(4)}$, $q^{(5)}$ we observe that
    \begin{equation*} \label{eq:third_order_euler_lagrangian_h_with_parts}
      \frac{\partial L}{\partial q} -
      \frac{d}{dt} \biggl(
      \underbrace{
      \frac{\partial L}{\partial \dot{q}} -
      \frac{d}{dt} \biggl(
      \underbrace{
      \frac{\partial L}{\partial \ddot{q}} -
      \frac{d}{dt} \biggl(
      \overbrace{
      \frac{\partial L}{\partial q^{(3)}}
      }^{\text{depends on } q^{(3)}} \biggr)
      }_{\text{depends on } q^{(4)}} \biggr)
      }_{\text{depends on } q^{(5)}} \biggr)
      = 0
    \end{equation*}

    So it is convenient to define
    \begin{equation*}\label{eq:def_P}
      \begin{cases}
        P_3 := \text{part that depends on } q^{(3)}
             = \frac{\partial L}{\partial q^{(3)}} \\
        P_2 := \text{part that depends on } q^{(4)}
             = \frac{\partial L}{\partial \ddot{q}} -
               \frac{d}{dt} \left(
                 \frac{\partial L}{\partial q^{(3)}}
               \right) \\
        P_1 := \text{part that depends on } q^{(5)}
             = \frac{\partial L}{\partial \dot{q}} -
               \frac{d}{dt} \left(
                 \frac{\partial L}{\partial \ddot{q}} -
                 \frac{d}{dt} \left(
                   \frac{\partial L}{\partial q^{(3)}}
                  \right)
                \right) \\
      \end{cases}
    \end{equation*}
  \end{alertblock}
\end{frame}

\begin{frame}{Motivation behind the definition of canonical variables}
  \begin{alertblock}{}
    Definition for $P_i$ can be rewritten in a more compact notation employing
    the \emph{functional derivative}
    \begin{equation*}
      \begin{cases}
        P_3  = \frac{\partial L}{\partial q^{(3)}}
             = \frac{\delta L}{\delta q^{(3)}} \\
        P_2  = \frac{\partial L}{\partial \ddot{q}} -
               \frac{d}{dt} \left(
                 \frac{\partial L}{\partial q^{(3)}}
               \right)
             = \frac{\delta L}{\delta \ddot{q}} \\
        P_1  = \frac{\partial L}{\partial \dot{q}} -
               \frac{d}{dt} \left(
                 \frac{\partial L}{\partial \ddot{q}} -
                 \frac{d}{dt} \left(
                   \frac{\partial L}{\partial q^{(3)}}
                  \right)
                \right)
             = \frac{\delta L}{\delta \dot{q}} \\
      \end{cases}
    \end{equation*}

    Generalizing to the nth-order Lagrangian we obtain the previous definition
    of canonical variables
    \begin{equation*}
      Q_{(i)} := q^{(i-1)} = \frac{d^{\, i-1} q}{dt^{\, i-1}}
      \quad \leftrightarrow \quad
      P_{(i)} := \frac{\delta L}{\delta q^{(i)}}
      \qquad i = 1, \ldots, n
    \end{equation*}
  \end{alertblock}
\end{frame}
