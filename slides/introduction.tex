\begin{frame}{Why study HD systems?}
  \begin{columns}
    \begin{column}{0.5\textwidth}
      \begin{figure}
        \includegraphics[height=50]{./figures/quantum-field-theory.tikz}
        \caption[labelformat=empty]{Quantum Field Theory (QFT)}\label{fig:QFT}
      \end{figure}
    \end{column}
    \begin{column}{0.5\textwidth}
      \begin{figure}
        \includegraphics[height=50]{./figures/general-relativity.tikz}
        \caption[labelformat=empty]{General Relativity (GR)}\label{fig:GR}
      \end{figure}
    \end{column}
  \end{columns}
  \vspace{1em}
  \begin{center}
    \emph{Desideratum}: the whole description of Universe, \\
    a \alert{Theory of everything}.
  \end{center}
\end{frame}

\begin{frame}{Why study HD systems?}
  \begin{alertblock}{QFT + GR = Theory of everything}
      \vspace{0.5em}
      This is what string theory try to achieve but some problems arose:
      \begin{itemize}
        \item does not provide us a complete description of Nature
        \item no known doable experiments to corroborate it
      \end{itemize}
  \end{alertblock}
  So other path may be explored in order to formulate a Theory of everything.
\end{frame}

\begin{frame}{Why study HD systems?}
  \begin{alertblock}{Ordinary Field Theory = Theory of everything}
    \vspace{0.5em}
  %\begin{columns}
    %\begin{column}{0.45\textwidth}
      Explain the curvature of spacetime by thinking our universe as (3+1)
      dimension thin film in a flat higher-dimensional space.
    %\end{column}
    %\begin{column}{0.45\textwidth}
    %  \begin{figure}
    %    \includegraphics[height=90]{./figures/universe.tikz}
    %    \caption[labelformat=empty]{bulk ---
    %    \alert{universe}}\label{fig:Universe}
    %  \end{figure}
    %\end{column}
  %\end{columns}
  \end{alertblock}
  In this field theory in the \emph{Lagrangian} appears \alert{higher
  derivative terms}~\cite{Smilga17}.
\end{frame}
