% !TEX program = xelatex

\documentclass[10pt]{beamer}
\usetheme{metropolis}
\usepackage{appendixnumberbeamer}


% -----------------------------------Packages--------------------------------- %

% Set the language of the document (e.g. title, section, abstract, …)
\usepackage[english]{babel}


% Finer control on caption formatting
\usepackage{caption}

% For importing tikz figures
\usepackage{tikzscale}


% --------------------------------Caption config------------------------------ %

\captionsetup[figure]{labelformat=empty}


% ---------------------------------Tkiz config-------------------------------- %

\usetikzlibrary{backgrounds}
\usetikzlibrary{arrows}
\usetikzlibrary{shapes,shapes.geometric,shapes.misc}

% this style is applied by default to any tikzpicture included via \tikzfig
\tikzstyle{tikzfig}=[baseline=-0.25em,scale=0.5]

% these are dummy properties used by TikZiT, but ignored by LaTex
\pgfkeys{/tikz/tikzit fill/.initial=0}
\pgfkeys{/tikz/tikzit draw/.initial=0}
\pgfkeys{/tikz/tikzit shape/.initial=0}
\pgfkeys{/tikz/tikzit category/.initial=0}

% standard layers used in .tikz files
\pgfdeclarelayer{edgelayer}
\pgfdeclarelayer{nodelayer}
\pgfsetlayers{background,edgelayer,nodelayer,main}

% style for blank nodes
\tikzstyle{none}=[inner sep=0mm]


% fix strange self-loops, which are PGF/TikZ default
\tikzstyle{every loop}=[]

% import tikzstyle
\input{metropolis.tikzstyles}


% ----------------------------------Document---------------------------------- %

\title{Higher-Order Lagrangians in classical and quantum systems}
\date{\today}
\author{Simone Bertolotto\\{\small Supervisor: Prof.\ Igor Pesando}}
%\author{Simone Bertolotto}
%\author[Supervisor]{Prof.\ Igor Pesando}
\institute{Università degli studi di Torino --- Fisica}

\begin{document}

  \maketitle

  \section{Introduction}

  \begin{frame}{Why study HD systems?}
    \begin{columns}
      \begin{column}{0.5\textwidth}
        \begin{figure}
          \includegraphics[height=50]{./figures/quantum-field-theory.tikz}
          \caption[labelformat=empty]{Quantum Field Theory (QFT)}\label{fig:QFT}
        \end{figure}
      \end{column}
      \begin{column}{0.5\textwidth}
        \begin{figure}
          \includegraphics[height=50]{./figures/general-relativity.tikz}
          \caption[labelformat=empty]{General Relativity (GR)}\label{fig:GR}
        \end{figure}
      \end{column}
    \end{columns}
    \vspace{1em}
    \begin{center}
      \emph{Desideratum}: the whole description of Universe, \\
      a \alert{Theory of everything}.
    \end{center}
  \end{frame}

  \begin{frame}{Why study HD systems?}
    \begin{alertblock}{QFT + GR = Theory of everything}
        \vspace{0.5em}
        This is what string theory try to achieve but some problems arose:
        \begin{itemize}
          \item does not provide us a complete description of Nature
          \item no known doable experiments to corroborate it
        \end{itemize}
    \end{alertblock}
    So other path may be explored in order to formulate a Theory of everything.
  \end{frame}

  \begin{frame}{Why study HD systems?}
    \begin{alertblock}{Ordinary Field Theory = Theory of everything}
      \vspace{0.5em}
    %\begin{columns}
      %\begin{column}{0.45\textwidth}
        Explain the curvature of spacetime by thinking our universe as (3+1)
        dimension thin film in a flat higher-dimensional space.
      %\end{column}
      %\begin{column}{0.45\textwidth}
      %  \begin{figure}
      %    \includegraphics[height=90]{./figures/universe.tikz}
      %    \caption[labelformat=empty]{bulk ---
      %    \alert{universe}}\label{fig:Universe}
      %  \end{figure}
      %\end{column}
    %\end{columns}
    \end{alertblock}
    In this field theory in the \emph{Lagrangian} appears \alert{higher
    derivative terms}~\cite{Smilga17}.
  \end{frame}

  \section{Lagrangian \& Hamiltonian formalism}

  \begin{frame}{Functional derivative}
    Define the \emph{action functional} as
    \begin{equation*}
      S[q] := \int_{t_1}^{t_2} dt \,
      L \left(q, \dot{q}, \ddot{q}, \ldots, q^{(n)} \right)
      \qquad \text{where} \qquad q^{(k)} := \frac{d^k q(t)}{dt^k}
    \end{equation*}

    Produce the \emph{k-th order perturbation on $q^{(k)}$}
    \begin{equation*}
      q^{(k)} \rightarrow \frac{d^k q(t)}{dt^k} + \frac{d^k \delta q(t)}{dt^k} =
      q^{(k)} + \delta q^{(k)}
    \end{equation*}

    and define the \emph{variation of the action functional} as
    \begin{equation*}
      \delta S[q] := S[q + \delta q] - S[q]
    \end{equation*}

    Considering $\delta q$ infinitesimal and keeping the ends points fixed we
    get % \\ (i.e. $\delta y^{k}(t_1) = \delta y^{k}(t_2) = 0$ for all k)
    \begin{equation*}
      \delta S[q] = \int_{t_1}^{t_2} dt \,
      \sum_{k=0}^{n} {(-1)}^k \frac{d^k}{dt^k}
      \left(\frac{\partial L}{\partial q^{(k)}}\right) \delta q
    \end{equation*}
  \end{frame}

  \begin{frame}{Functional derivative}
    Noting the similarities with the differential in multi-variable calculus
    \begin{equation*}
      df(x) = \sum_{k=1}^{n}
      \left\{\frac{\partial f(x)}{\partial x_k} dx_k\right\}
      \quad \longleftrightarrow \quad
      \delta S[q] = \int_{t_1}^{t_2} dt \,
      \left\{ \frac{\delta S[q]}{\delta q} \delta q \right\}
    \end{equation*}

    the \emph{functional derivative with respect to $q$} can be define
    \begin{equation*}
      \frac{\delta S[q]}{\delta q}:=
      \sum_{k=0}^{n} {(-1)}^k \frac{d^k}{dt^k}
      \left(\frac{\partial L}{\partial q^{(k)}}\right)
    \end{equation*}

    Generalization and abuse of notation lead to writing
    \metroset{block=fill}
    \begin{block}{Functional derivative with respect to $q^{(i)}$}
      \begin{equation*}
        \frac{\delta L}{\delta q^{(i)}}:=
        \sum_{k=0}^{n} {(-1)}^k \frac{d^k}{dt^k}
        \left(\frac{\partial L}{\partial q^{(i+k)}}\right)
      \end{equation*}
    \end{block}
  \end{frame}

  \begin{frame}{Lagrangian formalism}
    \begin{alertblock}{Hamilton's principle}
      \vspace{0.5em}
      The motion of the system is a stationary point of the action
      functional~\cite{Goldstein11_Ham_principle}
      \begin{equation*}
        \frac{\delta S[q]}{\delta q} = 0
      \end{equation*}
    \end{alertblock}
    \begin{alertblock}{Euler-Lagrange equation}
      \vspace{0.5em}
      \begin{equation*}
        \Rightarrow \qquad
        \frac{\delta S[q]}{\delta q} =
        \frac{\partial L}{\partial q} -
        \frac{d}{dt}\frac{\partial L}{\partial \dot{q}} +
        \frac{d^2}{dt^2}\frac{\partial L}{\partial \ddot{q}} -
        \ldots = 0
      \end{equation*}
    \end{alertblock}
  \end{frame}

  \begin{enumerate}
  \begin{frame}{Hamiltonian formalism}
    \begin{alertblock}{From Lagrange to Hamilton in 3 step}
      \vspace{0.5em}
      \item Define i-th \emph{canonical coordinates as}
        \begin{equation*} \label{eq:def_canonical_coordinates}
          Q_{(i)} := q^{(i-1)} = \frac{d^{\, i-1} q}{dt^{\, i-1}}
          \quad \leftrightarrow \quad
          P_{(i)} := \frac{\delta L}{\delta q^{(i)}}
          \qquad i = 1, \ldots, n
        \end{equation*}
      \item Define $\tilde{H}$ as
        \begin{equation*} \label{eq:Ham_in_q}
          \tilde{H}(q, \dot{q}, \ldots, q^{(n)}) :=
          \sum_{i=1}^{n} \frac{\delta L}{\delta q^{(i)}} q^{(i)} -
          L(q, \dot{q}, \ldots, q^{(n)})
        \end{equation*}
        Under the hypothesis of regular Lagrangian, relation for $q^{(n)}$ can
        be inverted
        \begin{equation*} \label{eq:qn=h}
          \frac{\delta L}{\delta q^{(n)}} = P_n
          \quad \Rightarrow \quad
          q^{(n)} = h(Q_{(1)}, \ldots, Q_{(n)}, P_{(n)}) \equiv h
        \end{equation*}
    \end{alertblock}
  \end{frame}

  \begin{frame}{Hamiltonian formalism}
    \begin{alertblock}{}
        Substituting in the expression for $\tilde{H}$ we get the
        \emph{Hamiltonian}
        \begin{equation*} \label{eq:general_hamiltonian}
          H = P_n h + P_{n-1} Q_n + \cdots +
              P_1 Q_2 - L(Q_1, Q_2, \ldots, h)
        \end{equation*}
      \item The Hamiltonian generates the time evolution of any function of
        canonical variables $F(Q_i, P_i)$ via the \emph{Poisson
        Brackets}~\cite{Chen13}
        \begin{equation*} \label{eq:poisson_bracket_evolution}
          \dot{F}(Q_i, P_i) = \left\{ F(Q_i, P_i), H \right\}
        \end{equation*}
        Poisson Brackets, in canonical coordinates, are defined as
        \begin{equation*} \label{eq:possion_braket}
          \left\{ f, g \right\} := \sum_{i=1}^{n} \left(
            \frac{\partial f}{\partial Q_{(i)}}
            \frac{\partial g}{\partial P_{(i)}} -
            \frac{\partial f}{\partial P_{(i)}}
            \frac{\partial g}{\partial Q_{(i)}}
          \right)
        \end{equation*}
    \end{alertblock}
  \end{frame}
  \end{enumerate}

  %\section{First order vs Second order Lagrangian}
  \section{First order vs Second order system}

  \begin{frame}{First order vs Second order Lagrangian}
    We would like to find the differences between systems that can be
    describe by these two Lagrangians:
    \vspace{0.5em}
    \begin{columns}
      \begin{column}{0.4\textwidth}
        \begin{center}
          \alert{First order system}
          \begin{equation*}
            L_1(q, \dot{q}) = \frac{\dot{q}^2}{2} - \alpha(q)
          \end{equation*}
        \end{center}
      \end{column}
      \begin{column}{0.4\textwidth}
        \begin{center}
          \alert{Second order system}
          \begin{equation*}
            L_2(q, \ddot{q}) = \frac{\ddot{q}^2}{2} - \alpha(q)
          \end{equation*}
        \end{center}
      \end{column}
    \end{columns}
  \end{frame}

  \begin{frame}{First order Lagrangian}
    \begin{equation*}
      L_1(q, \dot{q}) = \frac{\dot{q}^2}{2} - \alpha(q)
    \end{equation*}
    \vspace{0.2em}
    \begin{equation*} \label{eq: first-order_motion_eq_lagr}
      \frac{\partial L_1}{\partial q} -
      \frac{d}{dt}\frac{\partial L_1}{\partial \dot{q}} +
      \underbrace{
      \frac{d^2}{dt^2}\frac{\partial L_1}{\partial \ddot{q}} - \ldots }_0 = 0
      \qquad \Rightarrow \qquad
      \ddot{q} = - \frac{d\alpha(q)}{dq}
    \end{equation*}
    \begin{equation*}
      \begin{cases}
        Q_1 := q \\
        P_1 := \frac{\delta L_1}{\delta \dot{q}} = \dot{q}
      \end{cases}
      \qquad
      \tilde{H}_1(q, \dot{q}) :=
      \frac{\delta L_1}{\delta \dot{q}} \dot{q} - L_1(q, \dot{q})
    \end{equation*}
    \metroset{block=fill}
    \begin{block}{First order Hamiltonian}
      \begin{equation*} \label{eq: first-order_motion_eq_ham}
        H_1(Q_1, P_1) = \frac{P_1^2}{2} + \alpha(Q_1) \qquad
      \end{equation*}
    \end{block}
  \end{frame}

  \begin{frame}{Second order Lagrangian}
    \begin{equation*}
      L_2(q, \ddot{q}) = \frac{\ddot{q}^2}{2} - \alpha(q)
    \end{equation*}
    \vspace{0.2em}
    \begin{equation*} \label{eq: second-order_motion_eq_lagr}
      \frac{\partial L_2}{\partial q} -
      \frac{d}{dt}\frac{\partial L_2}{\partial \dot{q}} +
      \frac{d^2}{dt^2}\frac{\partial L_2}{\partial \ddot{q}} -
      \underbrace{
        \frac{d^3}{dt^3}\frac{\partial L_2}{\partial q^{(3)}} + \ldots
      }_0 = 0
      \quad \Rightarrow \quad
      q^{(4)} = - \frac{d\alpha(q)}{dq}
    \end{equation*}
    \begin{equation*}
      \begin{cases}
        Q_1 := q \\
        P_1 := \frac{\delta L_2}{\delta \dot{q}}
             = \frac{\partial L_2}{\partial \dot{q}} -
               \frac{d}{dt} \left( \frac{\partial L_2}{\partial \ddot{q}} \right)
             = 0 - \frac{d}{dt} \left( \ddot{q} \right) = -q^{(3)}
      \end{cases}
      \begin{cases}
        Q_2 := \dot{q} \\
        P_2 := \frac{\delta L_2}{\delta \ddot{q}}
             = \frac{\partial L_2}{\partial \ddot{q}} = \ddot{q}
      \end{cases}
    \end{equation*}

    Therefore
    \begin{equation*}
      \begin{cases}
        q        = Q_1 \\
        \dot{q}  = Q_2 \\
        \ddot{q} = P_2
      \end{cases}
      \qquad
      \begin{cases}
        \frac{\delta L_2}{\delta \dot{q}}  = P_1 \\
        \frac{\delta L_2}{\delta \ddot{q}} = P_2
      \end{cases}
    \end{equation*}
  \end{frame}

  \begin{frame}
    \begin{equation}\label{eq:canonical_variable_second_order}
      \begin{cases}
        q        = Q_1 \\
        \dot{q}  = Q_2 \\
        \ddot{q} = P_2
      \end{cases}
      \qquad
      \begin{cases}
        \frac{\delta L_2}{\delta \dot{q}}  = P_1 \\
        \frac{\delta L_2}{\delta \ddot{q}} = P_2
      \end{cases}
    \end{equation}
    \begin{equation}\label{eq:h_tilda_second_order}
      \tilde{H}_2(q, \dot{q}) :=
        \frac{\delta L_2}{\delta \dot{q}} \dot{q} +
        \frac{\delta L_2}{\delta \ddot{q}} \ddot{q} -
        L_2(q, \ddot{q})
        = \frac{\delta L_2}{\delta \dot{q}} \dot{q} +
        \frac{\delta L_2}{\delta \ddot{q}} \ddot{q} -
        \frac{\ddot{q}^2}{2} + \alpha(q)
    \end{equation}

    Substituting~\eqref{eq:canonical_variable_second_order}
    in~\eqref{eq:h_tilda_second_order} we get the Hamiltonian
    \begin{align*} \label{eq: second-order_motion_eq_ham}
      H_2(Q_1, Q_2, P_1, P_2)
        &= P_1 Q_2 + P_2 P_2 - \frac{P_2^2}{2} + \alpha(Q_1) \\
        &= P_1 Q_2 + \frac{P_2^2}{2} + \alpha(Q_1)
    \end{align*}

    \metroset{block=fill}
    \begin{block}{Second order Hamiltonian}
      \begin{equation*} \label{eq: second-order_motion_eq_ham}
        H_2(Q_1, Q_2, P_1, P_2) = P_1Q_2 + \frac{P_2^2}{2} + \alpha(Q_1)
      \end{equation*}
    \end{block}
  \end{frame}

  \begin{frame}{First order vs Second order Hamiltonian}
    \vspace{0.5em}
    \begin{columns}
      \begin{column}{0.5\textwidth}
        \begin{center}
          \alert{First order system} \vspace{0.1em}
          \begin{equation*}
            L_1(q, \dot{q}) = \frac{\dot{q}^2}{2} - \alpha(q)
          \end{equation*}
          \begin{equation*} \label{eq: first-order_motion_eq_ham}
            H_1(Q_1, P_1) = \frac{P_1^2}{2} + \alpha(Q_1) \qquad
          \end{equation*}
        \end{center}
      \end{column}
      \begin{column}{0.5\textwidth}
        \begin{center}
          \alert{Second order system} \vspace{0.1em}
          \begin{equation*}
            L_2(q, \ddot{q}) = \frac{\ddot{q}^2}{2} - \alpha(q)
          \end{equation*}
          \begin{equation*} \label{eq: second-order_motion_eq_ham}
            H_2(Q_1, Q_2, P_1, P_2) = P_1Q_2 + \frac{P_2^2}{2} + \alpha(Q_1)
          \end{equation*}
        \end{center}
      \end{column}
    \end{columns}
    \vspace{2.0em}
    There is a significant difference between the two spectra of $H_1$ and
    $H_2$: the first is bounded from below while the latter is not (and both are
    not bounded from above).
  \end{frame}

  \section{Linear Ostrogradskian instability}

  \begin{frame}{Linear Ostrogradskian instability}
    \begin{alertblock}{$H_2$ system isolated}
      \vspace{0.5em}
      If the energy is conserve even though the spectra is not bounded the
      energy stay constant.
    \end{alertblock}
    \vspace{2.0em}
    \begin{alertblock}{$H_2$ interacting with $H_1$}
      \vspace{0.5em}
      $H_2$ system try to reach the minimum of the Hamiltonian $H_2$ by giving
      energy to $H_1$ system. This is behaviour goes on endlessly. This is the
      so called \alert{Linear Ostrogradskian instability}~\cite{Kallosh08,
      Eliezer89}.
    \end{alertblock}
  \end{frame}

  \begin{frame}{Ostrogradsky Theorem}
    \begin{theorem}[Ostrogradsky] \vspace{0.5em}
      If in a second (or higher) order Lagrangian the canonical momentum $P_n$
      does not vanish, the corresponding Hamiltonian may acquire an arbitrary
      real value~\cite{Smilga17}.
    \end{theorem}
  \end{frame}

  \begin{frame}{Quantum counterpart}
    In quantum mechanics \emph{linear Ostrogradskian instability} goes under the
    name ``\emph{ghost}''. Just like classical HD theories, quantum HD theories
    have the spectrum of the Hamiltonian unbounded. The previous Theorem have
    its quantum counterpart:
    \vspace{2.0em}
    \begin{theorem} \vspace{0.5em}
      The quantum counterpart of Hamiltonian derived from higher order
      Lagrangian of a non-degenerate higher-derivative system has no ground
      state~\cite{Smilga17}.
    \end{theorem}
  \end{frame}

  \begin{frame}{Curing linear Ostrogradskian instability with constraints}
    As shown in~\cite{Chen13} linear Ostrogradskian instability can be cured by
    imposition of constrains on the system. This can happen when the constrained
    Hamiltonian lives in phase space with lower dimensionality than the original
    phase space. \vspace{1.0em}
    \begin{block}{Example}
      \vspace{0.2em}
      \begin{equation*}
        L =
        \underbrace{
          \frac{1}{2} \left[
          \ddot{q}^2 - (\omega_1^2 + \omega_2^2) \dot{q}^2 +
          \omega_1^2 \omega_2^2 q^2 \right]
        }_{\text{Lagrangian of the system}} +
        \underbrace{
          4 \omega_1^2\omega_2^2 q^2 \lambda (1+ \lambda) +
          2 \sqrt{2} \omega_1\omega_2 \lambda q \ddot{q}
        }_{\text{constraint}}
      \end{equation*}
      \vspace{0.2em}
      \begin{equation*}
        H =\ \frac{\omega_1^2\omega_2^2}{2} Q_1^2 +
        \frac{\omega_1\omega_2}{\sqrt{2} {\left(\sqrt{2} \omega_1\omega_2 -
        \omega_1^2 - \omega_2^2 \right)}^2} P_1^2
      \end{equation*}
    \end{block}
  \end{frame}

  \section{Conclusions}

  \begin{frame}{Conclusions}
    \begin{itemize}
      \item Higher order Lagrangians emerge from field theories formulated in
        higher dimensional bulk. \vspace{0.5em}
      \item Most of higher order theories suffer of \emph{linear Ostrogradskian
        instability} and due to the consequences of \emph{Ostrogradsky Theorem}
        they have been discard a priori. \vspace{0.5em}
      \item One way to cure Ostrogradskian instability is by imposing
        constraints.
    \end{itemize} \vspace{1.0em}
    $\Rightarrow$ higher order theories should be investigated more
    deeply with particular attention to the energy spectrum.
  \end{frame}

  \begin{frame}[standout]
    Thank you!
  \end{frame}

  \begin{frame}[allowframebreaks]{References}
    \bibliography{../document/references}
    \bibliographystyle{abbrv}
  \end{frame}

\end{document}
