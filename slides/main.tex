% !TEX program = xelatex

\documentclass[10pt]{beamer}
\usetheme{metropolis}


% -----------------------------------Packages--------------------------------- %

% Set the language of the document (e.g. title, section, abstract, …)
\usepackage[english]{babel}


% Finer control on caption formatting
\usepackage{caption}

% For importing tikz figures
\usepackage{tikzscale}


% --------------------------------Caption config------------------------------ %

\captionsetup[figure]{labelformat=empty}


% ---------------------------------Tkiz config-------------------------------- %

\usetikzlibrary{backgrounds}
\usetikzlibrary{arrows}
\usetikzlibrary{shapes,shapes.geometric,shapes.misc}

% this style is applied by default to any tikzpicture included via \tikzfig
\tikzstyle{tikzfig}=[baseline=-0.25em,scale=0.5]

% these are dummy properties used by TikZiT, but ignored by LaTex
\pgfkeys{/tikz/tikzit fill/.initial=0}
\pgfkeys{/tikz/tikzit draw/.initial=0}
\pgfkeys{/tikz/tikzit shape/.initial=0}
\pgfkeys{/tikz/tikzit category/.initial=0}

% standard layers used in .tikz files
\pgfdeclarelayer{edgelayer}
\pgfdeclarelayer{nodelayer}
\pgfsetlayers{background,edgelayer,nodelayer,main}

% style for blank nodes
\tikzstyle{none}=[inner sep=0mm]


% fix strange self-loops, which are PGF/TikZ default
\tikzstyle{every loop}=[]

% import tikzstyle
\input{metropolis.tikzstyles}


% ----------------------------------Document---------------------------------- %

\title{TODO}
\date{\today}
\author{Simone Bertolotto}
\institute{Università degli studi di Torino --- Fisica}

\begin{document}

  \maketitle

  \section{Introduction}

  \begin{frame}{Why study HD systems?}
    \begin{columns}
      \begin{column}{0.5\textwidth}
        \begin{figure}
          \includegraphics[height=50]{./figures/quantum-field-theory.tikz}
          \caption[labelformat=empty]{Quantum Field Theory (QFT)}\label{fig:QFT}
        \end{figure}
      \end{column}
      \begin{column}{0.5\textwidth}
        \begin{figure}
          \includegraphics[height=50]{./figures/general-relativity.tikz}
          \caption[labelformat=empty]{General Relativity (GR)}\label{fig:GR}
        \end{figure}
      \end{column}
    \end{columns}
    \vspace{1em}
    \begin{center}
      \emph{Desideratum}: the whole description of Universe, \\
      a \alert{Theory of everything}.
    \end{center}
  \end{frame}

  \begin{frame}{Why study HD systems?}
    \begin{alertblock}{QFT + GR = Theory of everything}
        \vspace{0.5em}
        This is what string theory try to achieve but some problems arose:
        \begin{itemize}
          \item does not provide us a complete description of Nature
          \item no known doable experiments to corroborate it
        \end{itemize}
    \end{alertblock}
    So other path may be explored in order to formulate a Theory of everything.
  \end{frame}

  \begin{frame}{Why study HD systems?}
    \begin{alertblock}{Ordinary Field Theory = Theory of everything}
    \begin{columns}
      \begin{column}{0.45\textwidth}
        Explain the curvature of spacetime by thinking our universe as (3+1)
        dimension thin film in a flat higher-dimensional space.
      \end{column}
      \begin{column}{0.45\textwidth}
        \begin{figure}
          \includegraphics[height=90]{./figures/universe.tikz}
          \caption[labelformat=empty]{bulk ---
          \alert{universe}}\label{fig:Universe}
        \end{figure}
      \end{column}
    \end{columns}
    \end{alertblock}
    In this fields theory in the \emph{Lagrangian} appears \alert{higher
    derivative terms}.
  \end{frame}

  \begin{frame}[standout]
    Thank you!
  \end{frame}

\end{document}
