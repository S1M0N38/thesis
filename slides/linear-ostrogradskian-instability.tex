\begin{frame}{Linear Ostrogradskian instability}
  \begin{alertblock}{$H_2$ system isolated}
    \vspace{0.5em}
    If the energy is conserved even though the spectra is not bounded the
    energy stays constant.
  \end{alertblock}
  \vspace{2.0em}
  \begin{alertblock}{$H_2$ interacting with $H_1$}
    \vspace{0.5em}
    $H_2$ system tries to reach the minimum of the Hamiltonian $H_2$ by giving
    energy to $H_1$ system. This is behaviour goes on endlessly. This is the
    so called \alert{Linear Ostrogradskian instability}~\cite{Kallosh08,
    Eliezer89}.
  \end{alertblock}
\end{frame}

\begin{frame}{Ostrogradsky Theorem}
  \begin{theorem}[Ostrogradsky] \vspace{0.5em}
    If in a second (or higher) order Lagrangian the canonical momentum $P_n$
    does not vanish, the corresponding Hamiltonian may acquire an arbitrary
    real value~\cite{Smilga17}.
  \end{theorem}
\end{frame}

\begin{frame}{Curing linear Ostrogradskian instability with constraints}
  As shown in~\cite{Chen13} the linear Ostrogradskian instability can be cured
  by imposition of constrains on the system. This means that the constrained
  Hamiltonian lives in phase space with lower dimensionality than the original
  phase space.\\\vspace{0.3em}
  Constraints may be introduced in the Lagrangian using auxiliary variables
  $\lambda$, then they can be studied using the Dirac formalism for
  constraints.\vspace{0.7em}

  \begin{block}{Example: classical Pais-Uhlenbeck oscillator}
    \vspace{0.2em}
    \begin{equation*}
      L =
      \underbrace{
        \frac{1}{2} \left[
        \ddot{q}^2 - (\omega_1^2 + \omega_2^2) \dot{q}^2 +
        \omega_1^2 \omega_2^2 q^2 \right]
      }_{\text{Lagrangian of the system}} +
      \underbrace{
        4 \omega_1^2\omega_2^2 q^2 \lambda (1+ \lambda) +
        2 \sqrt{2} \omega_1\omega_2 \lambda q \ddot{q}
      }_{\text{constraint}}
    \end{equation*}
    \vspace{0.2em}
    \begin{equation*}
      H =\ \frac{\omega_1^2\omega_2^2}{2} Q_1^2 +
      \frac{\omega_1\omega_2}{\sqrt{2} {\left(\sqrt{2} \omega_1\omega_2 -
      \omega_1^2 - \omega_2^2 \right)}^2} P_1^2
    \end{equation*}
  \end{block}
\end{frame}


\begin{frame}{Quantum counterpart}
  In quantum mechanics \emph{linear Ostrogradskian instability} implies the
  existence of ``\emph{ghosts}''. Just like classical HD theories, quantum HD
  theories have the spectrum of the Hamiltonian unbounded. The previous Theorem
  have its quantum counterpart:
  \vspace{2.0em}
  \begin{theorem} \vspace{0.5em}
    The quantum counterpart of Hamiltonian derived from higher order
    Lagrangian of a non-degenerate higher-derivative system has no ground
    state~\cite{Smilga17}.
  \end{theorem}
\end{frame}

\begin{frame}{Quantum Pais-Uhlenbeck oscillator}
  \begin{block}{Example: quantum Pais-Uhlenbeck oscillator}
    \vspace{0.2em}
    \begin{equation*}
      \hat{H} = \hat{p}_x \hat{v} + \frac{\hat{p}_v^2}{2}
      + \left(\omega_1^2 + \omega_2^2\right)\frac{\hat{v}^2}{2}
      - \omega_1^2\omega_2^2 \frac{\hat{x}^2}{2}
    \end{equation*}
    where the quantum operators act on $\psi(x, v)$ as follows
    \begin{align*}
      \hat{x}\psi(x, v) &= x \psi(x, v) &
      \hat{v}\psi(x, v) &= v \psi(x, v) \\
      \hat{p}_x\psi(x, v) &= -i\partial_x \psi(x, v) &
      \hat{p}_v\psi(x, v) &= -i\partial_v \psi(x, v)
    \end{align*}
    Performing this quantum canonical transformation~\cite{Mannheim05}
    \begin{align*}
      \hat{X}_1 &= \frac{1}{\omega_1}
        \frac{\hat{p}_x \, + \,  \omega_1^2 v} {\sqrt{\omega_1^2-\omega_2^2}} &
      \hat{P}_1 & \equiv -i \partial_{X_1} = \omega_1
        \frac{\hat{p}_v \, + \,  \omega_2^2 x} {\sqrt{\omega_1^2-\omega_2^2}} \\
      \hat{X}_2 &=
        \frac{\hat{p}_v \, + \,  \omega_1^2 x} {\sqrt{\omega_1^2-\omega_2^2}} &
      \hat{P}_2 & \equiv -i \partial_{X_2} =
        \frac{\hat{p}_x \, + \,  \omega_2^2 v} {\sqrt{\omega_1^2-\omega_2^2}}
    \end{align*}
  \end{block}
\end{frame}

\begin{frame}{Quantum Pais-Uhlenbeck oscillator}
  we obtain
  \begin{equation*}
    \hat{H} =
    \frac{\hat{P}_1^2 \, + \, \omega_1^2 \hat{X}_1^2}{2} -
    \frac{\hat{P}_2^2 \, + \, \omega_2^2 \hat{X}_2^2}{2} \equiv
    \hat{H}_1 - \hat{H}_2
  \end{equation*}
  which is the difference of two independent quantum harmonic oscillators.
  \vspace{1.5em}
  \metroset{block=fill}
  \begin{block}{Energy spectrum}
    \begin{equation*}
      E_{mn} =
      \omega_1 \left( \frac{1}{2} + n \right) -
      \omega_2 \left( \frac{1}{2} + m \right)
      \qquad m,n = 0,1,2,\ldots
    \end{equation*}
  \end{block}
\end{frame}

\begin{frame}{Ghosts}
  Ghosts (i.e quantum states with negatives norms) appear if we try to get rid
  of negatives energies.

  First rewrite the Hamiltonian operator using
  \begin{equation*}
    \hat{a}_j := \frac{1}{\sqrt{2\omega_j}}
      \left(\partial_{X_j} + \omega_j X_j\right) \quad
    \hat{a}_j^{\dagger} := \frac{1}{\sqrt{2\omega_j}}
      \left(-\partial_{X_j} + \omega_j X_j\right)
    \quad \text{where} \quad
    \left[ \hat{a}_j, \hat{a}_j^{\dagger} \right] = 1
  \end{equation*}
  \begin{align*}
    \hat{H} &=
      \omega_1 \left( \hat{a}_1^{\dagger} \hat{a}_1 + \frac{1}{2} \right) -
      \omega_2 \left( \hat{a}_2 \hat{a}_2^{\dagger} - \frac{1}{2} \right) \\
            &=
      \omega_1 \hat{a}_1^{\dagger} \hat{a}_1 -
      \omega_2 \hat{a}_2 \hat{a}_2^{\dagger} +
      \frac{\omega_2 + \omega_1}{2}
  \end{align*}

  Notice that the second term produces excited states with decreasing negatives
  energies. We can stop this behavior by \emph{formally} defining a state
  $\phi(X_2)$ such that $\hat{a}_2^{\dagger} \ \phi(X_2) = 0$
\end{frame}

\begin{frame}{Ghosts}
  \begin{alertblock}{Properties of $\phi$}
    \begin{itemize}
      \item $\phi(X_2) \propto \exp \left\{ \frac{\omega_2}{2} X_2^2\right\}$
      \item its norm is positive.
      \item is not normalizable.
      \item formally has the lowest energy.
      \item $\hat{a}_2$ acts on it as creation operator.
      \item $\hat{a}_2^{\dagger}$ acts on it as annihilation operator.
    \end{itemize}
  \end{alertblock}

  $\Rightarrow$ Excited states of $\phi$ have negatives norms assuming
  $\bra{\phi} \ket{\phi} > 0$
  \begin{equation*}
    {\Big| \hat{a}_2 \ket{\phi} \Big|}^2 =
    \bra{\phi} \hat{a}_2^{\dagger} \hat{a}_2 \ket{\phi} =
    \underbrace{
      \bra{\phi} \hat{a}_2 \hat{a}_2^{\dagger} \ket{\phi}
    }_{0} - \bra{\phi} \ket{\phi} < 0
  \end{equation*}
\end{frame}
