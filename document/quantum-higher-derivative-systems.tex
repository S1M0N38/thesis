In this section the quantum counterpart of the previous classical system will be
studied. In the following, the process of \emph{canonical quantization} is used
to go from the classical world to the quantum one and a brief summary is given
in the Appendix~\ref{appendix:canonical_quantization}. This procedures will be
used to derive the Quantum Pais-Uhlenbeck (QPU) oscillator starting
from~\eqref{eq:lagrangian_PU}. The quantum version of the Ostrogradsky theorem
will be proved and lastly more example of ghost-ridden theories are given.

\subsection{Ghosts}\label{section:ghosts}
As said previously in Section~\ref{subsection:
linear_ostrogradskian_instability}, in quantum mechanics \emph{linear
Ostrogradskian instability} goes under the name ``\emph{ghost}''.  Just like
classical HD theories, quantum HD theories have the spectrum of the Hamiltonian
unbounded. Theorem~\ref{th:ostrogradsky_classical} have its quantum counterpart:

\begin{theorem}\label{th:ostrogradsky_quantum}~\cite{Smilga17}
  The quantum counterpart of Hamiltonian~\eqref{eq:general_hamiltonian_1dim}
  for a non-degenerate higher-derivative system has no ground state.
\end{theorem}
\begin{proof}
  The proof is given for a one-dimensional system ($\mu=1$) describe by a second
  order Lagrangian ($n=2$); the generalization for a multi-dimensional HD system
  is straightforward. The result Hamiltonian is
  \begin{equation*}
    H = h(Q_1, Q_2, P_2) P_2 + Q_2 P_1 - L(Q_1, Q_2, h)
  \end{equation*}
  Using the canonical quantization the function H is converted to a quantum
  observable $\hat{H}$ which eigenvalue are the allowed energies of the system.
  \begin{align*}
    & \begin{cases}
      Q_1 \xrightarrow{\wedge} \hat{Q_1} \equiv \hat{x}
      \quad &\mid \quad \hat{x}\psi(x, v) = x \psi(x, v) \\
      P_1 \xrightarrow{\wedge} \hat{P_1} \equiv \hat{p}_x
      \quad &\mid \quad \hat{p}_x\psi(x, v) = -i\partial_x \psi(x, v) \\
    \end{cases} \\
    & \begin{cases}
      Q_2 \xrightarrow{\wedge} \hat{Q_2} \equiv \hat{v}
      \quad &\mid \quad \hat{v}\psi(x, v) = v \psi(x, v) \\
      P_2 \xrightarrow{\wedge} \hat{P_2} \equiv \hat{p}_v
      \quad &\mid \quad \hat{p}_v\psi(x, v) = -i\partial_v \psi(x, v) \\
    \end{cases}
  \end{align*}
  \begin{equation} \label{eq:ham_operator_2nd_order}
    \hat{H} = h(\hat{x}, \hat{v}, \hat{p}_v) \hat{p}_v + \hat{v} \hat{p}_x -
    L (\hat{x}, \hat{v}, h)
  \end{equation}

  Moreover suppose that the spectrum of the Hamiltonian is discrete. If was
  continuous, normalize it in a box so the range of $x$ and $v$ is finite which
  make the spectrum discrete. At the end remove regularization sending the box
  size to infinity.
  Let $\psi_0(x, v)$ be the normalized ground state of the system with the
  associated eigenvalue $E_0$, i.e. $\hat{H}\psi_0(x, v) = E_0 \psi_0(x, v)$.
  To seek for an energy value close to $E_0$ we use the \emph{variational
  method}~\footnote{
    Variational methods must not be confused with variational calculus, they
    are totally different things.
  } with the following variational ansatz:

  \begin{equation}
    \psi_c(x, v) = e^{icx}\psi_0(x,v)
  \end{equation}

  Observe that $\psi_c(x,v)$ is not an eigenfunction of the
  Hamiltonian~\eqref{eq:ham_operator_2nd_order} and neither linear combination
  $\psi_f(x,v)$ of $\psi_c(x,v)$ are.

  \begin{align*}
    &\hat{H} \psi_c(x,v) = E_0\psi_c(x, v) + cv \psi_c(x,v)
    \neq \lambda \psi_c(x,v) \\ \\
    &\psi_f = \int f(c) \psi_c \, dc = \psi_0 \int f(c) e^{icx} \, dc =
    \psi_0 \tilde{f}(x) \\
    &\hat{H} \psi_f(x,v) = E_0\psi_f(x, v) -
    iv \frac{\partial \tilde{f}(x)}{\partial x} \psi_c(x,v)
    \neq \lambda \psi_c(x,v)
  \end{align*}

  So $\psi_c$ and $\psi_f$ does not vanish of exited state and variational
  energy must not exceed $E_0$ and can be obtain by direct calculation.

  \begin{equation*}
    E_0 \leq E_c = \int \psi_c^* \hat{H} \psi_c \, dx \, dv =
    E_0 + c \int v \left|\psi_0 \right| ^2 \, dx \, dv
  \end{equation*}

  If the integral $\int v \left|\psi_0 \right| ^2 \, dx \, dv$ does not vanish
  arbitrarily negative energies can be reach by an appropriate choice of $c$.
  Otherwise if the integral vanish, $E_c$ does not depend on $c$ and does not
  grow, as it should if $\psi_0$ would be a ground state~\cite{Smilga17}. Thus,
  $\psi_0$ is not a ground state as it was postulated. \\
\end{proof}


\subsection{Quantum Pais-Uhlenbeck oscillator}
The first study to be publish about a quantum HD theory was the \emph{Quantum
Pais-Uhlenbeck oscillator} (QPU)~\cite{PU50}. In order to obtain the Hamiltonian
(operator) $\hat{H}_{QPU}$ of the QPU system,
the $L_{PU}$~\eqref{eq:lagrangian_PU} had to be converted in the corresponding
Hamiltonian $H_{PU}$ and then canonically quantize it, i.e.
$L_{PU} \rightarrow H_{PU} \xrightarrow{\wedge} \hat{H}_{QPU}$
\footnote{
  Notation for quantum operators is
  the same used in the proof of Theorem~\ref{th:ostrogradsky_quantum}
}.

\begin{equation} \label{eq:hamiltonian_QPU}
  \hat{H}_{QPU} = \hat{p}_x \hat{v} + \frac{\hat{p}_v^2}{2}
  + \left(\omega_1^2 + \omega_2^2\right)\frac{\hat{v}^2}{2}
  - \omega_1^2\omega_2^2 \frac{\hat{x}^2}{2}
\end{equation}

\subsubsection{non-degenerate frequencies: $\omega_1 \neq \omega_2$}

Consider the following canonical transformation and then canonically quantize
the new canonical variables $\tilde{Q}$ and $\tilde{P}$
\footnote{
  The proposed transformation is canonical because the canonical commutation
  relations are still valid, i.e.
  $\left\{ \tilde{Q_i} \, , \, \tilde{Q_j}\right\} =
  \left\{ \tilde{P_i} \, , \, \tilde{P_j}\right\} = 0$ and
  $\left\{ \tilde{Q_i} \, , \, \tilde{P_j}\right\} = \delta_{ij}$
}~\cite{Mannheim05}.

\begin{align*}
  \tilde{Q_1} &= \frac{1}{\omega_1}
    \frac{P_1 \, + \,  \omega_1^2 Q_2} {\sqrt{\omega_1^2-\omega_2^2}}
              &\xrightarrow{\wedge}& &\hat{\tilde{Q}}_1 \equiv
  \hat{X}_1 &= \frac{1}{\omega_1}
    \frac{\hat{p}_x \, + \,  \omega_1^2 v} {\sqrt{\omega_1^2-\omega_2^2}}
    \\
  \tilde{P_1} &= \omega_1
    \frac{P_2 \, + \, \omega_2^2 Q_1} {\sqrt{\omega_1^2-\omega_2^2}}
              &\xrightarrow{\wedge}& &\hat{\tilde{P}}_1 \equiv
    \hat{P}_1 &= \omega_1
    \frac{\hat{p}_v \, + \,  \omega_2^2 x} {\sqrt{\omega_1^2-\omega_2^2}}
    \\
  \tilde{Q_2} &=
    \frac{P_2 \, + \,  \omega_1^2 Q_1} {\sqrt{\omega_1^2-\omega_2^2}}
              &\xrightarrow{\wedge}& &\hat{\tilde{Q}}_2 \equiv
  \hat{X}_2 &=
    \frac{\hat{p}_v \, + \,  \omega_1^2 x} {\sqrt{\omega_1^2-\omega_2^2}}
    \\
  \tilde{P_2} &=
    \frac{P_1 \, + \, \omega_2^2 Q_2} {\sqrt{\omega_1^2-\omega_2^2}}
              &\xrightarrow{\wedge}& &\hat{\tilde{P}}_2 \equiv
  \hat{P}_1 &=
    \frac{\hat{p}_x \, + \,  \omega_2^2 v} {\sqrt{\omega_1^2-\omega_2^2}}
\end{align*}

Using $\hat{X}$ and $\hat{P}$, the Hamiltonian~\eqref{eq:hamiltonian_QPU}
becomes the difference of two independent quantum harmonic oscillator.

\begin{equation}
  \hat{H} =
  \frac{\hat{P}_1^2 \, + \, \omega_1^2 \hat{X}_1^2}{2} -
  \frac{\hat{P}_2^2 \, + \, \omega_2^2 \hat{X}_2^2}{2} \equiv
  \hat{H}_1 - \hat{H}_2
\end{equation}

The spectrum of $\hat{H}$ follows from the solution of the equation
$\hat{H} \psi = E \psi$. Let $\psi$ be the product of the eigenstates of two
independent simple one-dimensional quantum harmonic oscillator $\psi_n$ and
$\psi_m$, i.e.

\begin{equation*}
  \psi(X_1, X_2) = \psi_n(X_1) \psi_m(X_2)
  \quad \text{where} \quad
  \begin{cases}
    &\hat{H}_1\psi_n = \omega_1 \left( \frac{1}{2} + n \right)\psi_n \\
    &\hat{H}_2\psi_m = \omega_2 \left( \frac{1}{2} + m \right)\psi_m
  \end{cases}
\end{equation*}

\begin{align*}
  \hat{H} \psi
  &= \left(\hat{H}_1 - \hat{H}_2\right) \left(\psi_n \psi_m\right) \\
  &= \left(\hat{H}_1\psi_n\right)\psi_m - \left(\hat{H}_2\psi_m\right)\psi_n \\
  &= \left(
    \omega_1\left(\frac{1}{2} + n\right) -
    \omega_2\left(\frac{1}{2} + m\right)
     \right) \left(\psi_n\psi_m\right)
\end{align*}



So the spectrum is

\begin{equation}
  E_{mn} =
  \omega_1 \left( \frac{1}{2} + n \right) -
  \omega_2 \left( \frac{1}{2} + m \right)
  \qquad m,n = 0,1,2,\ldots
\end{equation}

Consider the situation where ratio between $\omega_2$ and $\omega_1$ is a
rational number $k$; the spectrum can be written as

\begin{equation*}
  E_{mn} = \omega_1 \left( \frac{1-k}{2} + n + mk \right)
\end{equation*}

so the spectrum is discrete. On the other hand if the frequencies are
incommensurable the every real number can be hit by an appropriate choice of
$n$ and $m$. But the wave function of all eigenstate are normalizable, so the
spectrum is not continuous but rather pure point.


\subsubsection{degenerate frequencies: $\omega_1 = \omega_2$}
If frequencies $\omega_1$ and $\omega_2$ are the same, the previous
transformation can not be perform due to the vanishing denominator. Nevertheless
a solution can be obtain following the procedure describe in~\cite{PU50,
Bolonek06}. Let $\psi$ be written as

\begin{equation*}
  \psi(x, v) = e^{-i\omega^2xv}\phi(x,v)
\end{equation*}

So $\hat{H}$ acts on such $\psi$ as

\begin{equation*}
  \hat{H} \psi = e^{-i\omega^2xv}
  \left[\frac{\hat{p}_v^2}{2} + v\hat{p}_x - \omega^2x\hat{p}_v \right] \phi
  \quad \Rightarrow \quad
  \hat{H}_{\phi} :=
  \frac{\hat{p}_v^2}{2} + v\hat{p}_x - \omega^2x\hat{p}_v
\end{equation*}

Now, on $\hat{H}_{\phi}$ are apply a scale transformation and then a unitary
transformation\footnote{
  During the calculation of $\hat{H}'_{\phi}$ was useful swapping
  two non-commuting operators (e.g. $v$ and $e^{\lambda \hat{p}_v}$)
  so that their commutator appears. This identity was used:
  $\left[e^{\lambda \, \hat{A}}, \, \hat{B}\right] =
  \lambda \, e^{\lambda \, \hat{A}} \left[\hat{A} , \, \hat{B} \right]$
}.

\begin{equation*}
  \text{scale} \
  \begin{cases}
    &x         \rightarrow x/\omega \\
    &\hat{p}_x \rightarrow \omega\hat{p}_x
  \end{cases}
  \qquad \text{unitary} \;
  \U = \exp\left\{ - \frac{\hat{p}_x \hat{p}_v}{4}\right\}
\end{equation*}

\begin{equation*}
  \hat{H}'_{\phi} = \, \U^{\dagger}
  \hat{H}_{\phi} \left(x/\omega,\ v,\ \omega \hat{p}_x,\ \hat{p}_v\right)
  \ \U \qquad
  \phi'(x, v) = \U \phi\left(x/\omega, v\right)
\end{equation*}

\begin{equation} \label{eq:ham_phi_PU}
  \hat{H}'_{\phi} = \frac{\hat{p}^2_x \, + \, \hat{p}^2_{v}}{4}
  + \omega \left(v \hat{p}_x - x \hat{p}_v \right)
\end{equation}

Hamiltonian~\eqref{eq:ham_phi_PU} can be interpreted as the Hamiltonian of a
free particle plus the term $\omega \left(v \hat{p}_x - x \hat{p}_v \right)$
which resemble the orbital angular momentum in the $(\omega x, \, v)$ plane.
Going in polar coordinates and reverting back to the original $\psi$, we obtain

\begin{align} \label{eq:sol_ham_degen_PU}
  \psi_{l, \, k}(x, v \,; t) \ &\propto \
  e^{-i\omega^2xv}
  \exp\left\{
    \frac{i}{4\omega^2} \frac{\partial^2}{\partial x \partial v}
  \right\} \cdot \nonumber \\
  &\cdot \left[
    J_l \left(k\sqrt{v^2 + \omega^2 x^2}\right)
    {\left( \frac{wx - iv}{wx +iv} \right)}^{l/2}
  \right]
  e^{-it (l\omega + k^2/4)}
\end{align}

Wave functions~\eqref{eq:sol_ham_degen_PU} are not normalizable hence the
spectrum is continuous. Moreover each energy level infinitely degenerate:
$\psi_{l+1, \, k}$ and $\psi_{l, \, \sqrt{k^2 + 4 \omega}}$ have same
eigenvalue.
