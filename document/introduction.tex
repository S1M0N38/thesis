The ultimate goal of theoretical physics is the description of all natural
phenomena with a rigorous mathematical formalism. Natural phenomena can be
observe at large scale (planets, starts, galaxy, black hole, $\dots$),
at minuscule scale (molecules, atoms, particles, $\dots$) and in every day life
with our own naked eyes. Many agree on the fact that the ``invisible
hand'' that governs all such phenomena is one (the Nature) and follows the
same law independent of the scale. In this regard our description of the Nature
should be given in a manner that accommodate different scale phenomena in a
unify theory: a \emph{theory of everything} (TOE). \\

Nowadays the theory of \emph{General Relativity} describe the large scale
phenomena while the tiny ones are describe by \emph{Quantum Fields Theory} and
both theories are corroborated by numerous experimental evidence. Expert believe
that a TOE is a \emph{superstring theory} of some kind~\cite{Smilga17} but not
such a complete theory is known. Since its birth ($\approx 50$ year ago),
superstring theory has not yet provided us a complete understanding of the
Nature and, moreover, it have to deal with the fact that there are no known
doable experiments that can confirm or reject it. \\

So other roads may be explored in order to formulate a TOE\@. One of them starts
from the ansatz that TOE is an ordinary field theory. Even when proposing such a
different theory we have to be able to explain the experimental fact about the
curvature of spacetime. Unlike General Relativity we keep the space and time
flat and imagine our Universe (3+1)--dimensional as a thin film embedded in a
flat higher-dimensional bulk~\cite{Smilga17}.

\begin{displayquote}
  \emph{Our universe is a (3+1)--dimensional bubble soap in a flat
  higher-dimensional space.}
\end{displayquote}

Such a field theory is formulate in this higher-dimensional bulk and, in order
to still renormalizable, in the Lagrangian appears higher derivative
terms~\cite{Smilga17}. In light of this we have a good reason to study higher
derivative systems.
