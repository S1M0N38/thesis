Common lagrangian in introductory courses of classical mechanics depends of $q$
and $\dot{q}$. We refer to them as \emph{first-order lagrangian} because the
highest derivative term is $\dot{q}$, i.e. the first derivative of $q(t)$ with
respect to time. If in the lagrangian appears also $\ddot{q}$ it will be a
\emph{second-order lagrangian} and so on. \\

In the following the mathematical machinery, develop in section \ref{section:
lagrangin and hamiltonian formalism}, will be used to study two apparently
similar lagrangian and introduce the core problem of HD systems.

\subsection{First-order vs Second-order lagrangian}
The simplest non-trivial lagrangians are those in polynomial form like

\begin{equation*}
  L_1(q, \dot{q}) = \frac{\dot{q}^2}{2} - \alpha(q) \qquad
  L_2(q, \ddot{q}) = \frac{\ddot{q}^2}{2} - \alpha(q)
\end{equation*}

where $\alpha$ is a smooth function of $q$ only. No Greek indices appear on
$q^{(i)}$ because we limit ourself to one-dimensional systems. $L_1$ is a
first-order lagrangian while $L_2$ it's a second-order one.


\paragraph{first-order lagrangian} In order to get the equation of motion we
used the Euler-Lagrange equation \eqref{eq:euler-lagrange}

\begin{equation} \label{eq: first-order_motion_eq_lagr}
  \frac{\partial L_1}{\partial q} -
  \frac{d}{dt}\frac{\partial L_1}{\partial \dot{q}} +
  \underbrace{
  \frac{d^2}{dt^2}\frac{\partial L_1}{\partial \ddot{q}} - \ldots }_0 = 0
  \qquad \Rightarrow \qquad
  \ddot{q} = - \frac{d\alpha(q)}{dq}
\end{equation}

If $q$ is the position coordinate of a unit mass point, the function $\alpha(q)$
can be interpreted as the potential energy and \eqref{eq:
first-order_motion_eq_lagr} is nothing less than Newton's second law. To get the
hamiltonian description apply the recipe in section \ref{subsection:
hamiltonian_formalism}

\begin{equation*}
  \begin{cases}
    Q := q \\
    P := \frac{\delta L_1}{\delta \dot{q}} = \dot{q}
  \end{cases}
  \qquad
  H(q, \dot{q}) := \frac{\delta L_1}{\delta \dot{q}} \dot{q} - L_1(q, \dot{q})
\end{equation*}

\begin{equation} \label{eq: first-order_motion_eq_ham}
  H(Q, P) = \frac{P^2}{2} + \alpha(Q) \qquad
  \begin{cases}
    \dot{Q} =   \frac{\partial H}{\partial P} = P \\
    \dot{P} = - \frac{\partial H}{\partial Q} = - \frac{d\alpha(Q)}{dQ}
  \end{cases}
\end{equation}

Since $L_1$ does not explicitly depend on time (only through $q$ and $\dot{q}$),
the hamiltonian $H$ can be interpreted as the total energy of the system (sum of
kinetic and potential energy).  Combining the two equation in \eqref{eq:
first-order_motion_eq_ham} we restore \eqref{eq: first-order_motion_eq_lagr}.


\paragraph{second-order lagrangian} todo
