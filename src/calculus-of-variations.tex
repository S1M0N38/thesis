In this appendix are summarized some concepts and calculations of the
\emph{calculus of variations} extensively used in the paper
\footnote{
  Appendix \ref{appendix: calculus of variation} does not aim to be
  mathematically rigorous but expose fundamental ideas in a simple and concise
  way. For a proper introduction to this subject consult the relevant
  literature.
}.
Calculus of variations is the study how \emph{functionals} change under smalls
variations of their inputs.

\begin{definition}
  \cite{Functional_Encyclopedia_of_Mathematics}
  A functional is a $J$ map from an set $X$ into the set $\field{R}$ of real
  numbers or the set $\field{C}$ of complex numbers. If $X$ endowed with the
  structure of a vector space, a topological space or an ordered set, then
  there arise the important classes of linear, continuous and monotone
  functionals, respectively.
\end{definition}

Let $X:=\{y: [x_1, x_2] \subset \field{R} \rightarrow \field{R}: x \mapsto
y(x) \ | \ y(x) \in \mathcal{C}^{\infty}([x_1, x_2])\}$ and define the linear
continuous functional $J[y]$ as:

\begin{equation} \label{eq:def_functional}
  J[y] := \int_{x_1}^{x_2} f(y^{(0)}, y^{(1)}, y^{(2)}, \ldots, y^{(n)})
\end{equation}

A variation of the function $y(x)$ is denote by $\delta y(x)$ and correspond to
a generic small change of the function $y$ at $x$. Using the linearity of the
derivative with respect to $x$ we obtain the so called \emph{k-th order
perturbations}.

\begin{equation} \label{eq:def_function_variation}
  y^{(k)} \rightarrow \frac{d^ky(x)}{dx^k} + \frac{d^k\delta y(x)}{dx^k} =
  y^{(k)} + \delta y^{(k)} \qquad k=0, \ldots, n
\end{equation}

A variation of the functional $J[y]$ is denoted by $\delta J[y]$. It's given by
the total contribution of the variations of its inputs, i.e. $\delta y(x)$ and
it's define as

\begin{equation} \label{eq:def_functional_variation}
  \delta J[y] := J[y + \delta y] - J[y]
\end{equation}

Substituting \eqref{eq:def_function_variation} in
\eqref{eq:def_functional_variation} and expanding $J[y + \delta y]$ to first
order using Taylor expansion around $y$ we obtain

\begin{align*}
  \delta J[y]
  &=\int_{x_1}^{x_2} dx \left\{ f +
    \frac{\partial f}{\partial y^{(0)}} \delta y^{(0)} +
    \frac{\partial f}{\partial y^{(1)}} \delta y^{(1)} + \ldots +
    \frac{\partial f}{\partial y^{(n)}} \delta y^{(n)}
    \right\} - \int_{x_1}^{x_2} dx f \\
  &=\int_{x_1}^{x_2} dx
    \sum_{k = 0}^{n} \frac{\partial f}{\partial y^{(k)}} \delta y^{(k)}
   =\sum_{k = 0}^{n}
    \int_{x_1}^{x_2} dx \frac{\partial f}{\partial y^{(k)}} \delta y^{(k)} \\
\end{align*}

Using integration by part repeatedly the last term can be rewrite as

\begin{align}
  \label{eq:sum_temrs}
  \int_{x_1}^{x_2} dx \frac{\partial f}{\partial y^{(k)}} \delta y^{(k)} =
  & \sum_{j=0}^{k} (-1)^j \frac{d^j}{dx^j}
  \left( \frac{\partial f}{\partial y^{(k)}} \right) \delta y^{(k-1-j)}
  \Big|_{x_1}^{x_2} \\
  \label{eq:int_temrs}
  & + \int_{x_1}^{x_2} dx (-1)^k \frac{d^k}{dx^k}
  \left( \frac{\partial f}{\partial y^{(k)}}\right) \delta y
\end{align}

Usually in the application of calculus of variation ends points of $y(x)$ are
kept fixed and the every variation is zero, i.e.

\begin{equation} \label{eq:fixed_ends}
  \delta y^{(k)}(x_1) = \delta y^{(k)}(x_2) = 0  \qquad k = 0, \ldots, n
\end{equation}

Under this assumption the expression for $\delta J[y]$ drastically simplify and
the summation term in \eqref{eq:sum_temrs} vanish and remain only
\eqref{eq:int_temrs}. Substituting it in the original expression we get

\begin{equation} \label{eq:functional_variation_ends_fixed}
  \delta J[y] = \int_{x_1}^{x_2} dx \sum_{k = 0}^{n} (-1)^k \frac{d^k}{dx^k}
  \left( \frac{\partial f}{\partial y^{(k)}}\right) \delta y
\end{equation}

Equation \eqref{eq:functional_variation_ends_fixed} clearly resemble to the
expression for the differential of $f(x_1, x_2, \ldots, x_n)$ in multi-variable
calculus. One would wish to write a formal expression for the \emph{derivation
of the functional} with respect to the $y(x)$. This can be achieve by this
comparison

\begin{equation*}
  df(x) = \sum_{k=1}^{n}
  \left\{ \frac{\partial f(x)}{\partial x_k} dx_k \right\}
  \quad \longleftrightarrow \quad
  \delta J[y] = \int_{x_1}^{x_2} dx
  \left\{ \frac{\delta J[y]}{\delta y} \delta y \right\}
\end{equation*}

leading up to the notion of \emph{functional derivative}

\begin{equation} \label{eq:functional_derivative}
  \frac{\delta J[y]}{\delta y} := \sum_{k = 0}^{n} (-1)^k \frac{d^k}{dx^k}
  \left( \frac{\partial f}{\partial y^{(k)}}\right)
\end{equation}

Some \cite{Chen13}, abusing of notation,  write $\frac{\delta f}{\delta y}$
instead of $\frac{\delta J[y]}{\delta y}$ implying the connection
\eqref{eq:def_functional} between $J$ and $f$. This abusion it's also used in
this paper when defining the conjugate momenta
\eqref{eq:def_canonical_coordinates}.
