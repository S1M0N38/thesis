Before dive straight into the study of Higher Derivative (HD) systems, i.e.
systems characterized by lagrangian contains terms such as $\ddot{q}$,
${q}^{(3)}$, ..., it's better to review some basic concepts of the
\emph{lagrangian} and \emph{hamiltonian formalism} in order to fix ideas and
notation.

\subsection{Lagrangian formalism}
When talking about \emph{lagrangian formalism} we mean the whole mathematical
formalism that it's been develop around the \emph{lagrangian mechanics} and then
extends way beyond to the realm of classical mechanics. The lagrangian mechanics
was born as an alternative formulation of the \emph{newtonian mechanics}
describing a system analytically instead of the vector based approach used by
Newton. The keys concepts of lagrangian mechanics are: \emph{generalized
coordinates}, \emph{degrees of freedom} and \emph{constraints}.

\begin{definition}\label{def: generalized coordinates} \cite{Ginsberg08}
  \emph{Generalized coordinates} $q^{\mu}$ are a set of coordinates (i.e. a
  collection of real number) that describe uniquely the configuration of a
  system.
\end{definition}

\begin{definition}\label{def: dof} \cite{Ginsberg08}
  The minimum number $n$ of generalized coordinates required to specify the
  position of a system is the \emph{number of degrees of freedom} of that
  system.
\end{definition}

\begin{definition}\label{def: holonomic constraints}
  \cite{Goldstein11_hol_constraints}
  \emph{Holonomic constraints} are relation between generalized coordinates which can
  be expressed as:
  \begin{equation*}
      f(q^{\mu}) = 0 \qquad \mu = 1, \ldots, n
  \end{equation*}
\end{definition}

Once generalized coordinates has been chosen their evolution describe the
evolution of the system. According to classical mechanics $q^{\mu}$ evolve
leisurely in a smooth fashion following the \emph{equation of motion}. This
``smoothness'' allows following definition:

\begin{definition}
  The \emph{generalized velocities} $\dot{q}^{\mu}$ are the time derivatives of the
  generalized coordinates $q^{\mu}$, i.e.
  \begin{equation*}
    \dot{q}^{\mu} := \frac{dq^{\mu}(t)}{dt}
  \end{equation*}
  The same idea can used to define
  \begin{equation*}
    \ddot{q}^{\mu} := \frac{d^2q^{\mu}(t)}{dt^2} \qquad \ldots \qquad
        q^{(n)\mu} := \frac{d^nq^{\mu}(t)}{dt^n}
  \end{equation*}
\end{definition}

In newtonian mechanics the equations of motion are ordinary second order
differential equation related the position vector $\bm{x}$ and the forces
$\bm{F}$:

\begin{equation*}
  m \frac{d^2\bm{x}}{dt^2} = \bm{F}
\end{equation*}

Lagrangian mechanics do not start from equations of motion for the description
of the system; instead it's defined a function called the \emph{Lagrangian}:

\begin{equation} \label{eq:general_lagrangian}
  L(q^{\mu}, \dot{q}^{\mu}, \ddot{q}^{\mu}, \ldots, q^{(n)\mu})
\end{equation}

An \emph{action functional} $S[q^{\mu}]$ can be associated to a system described
by the lagrangian $L$ evolving from a position at time $t_1$ to another position
at time $t_2$

\begin{equation} \label{eq:general_action}
  S[q^{\mu}] := \int_{t_1}^{t_2} dt
  L(q^{\mu}, \dot{q}^{\mu}, \ddot{q}^{\mu}, \ldots, q^{(n)\mu})
\end{equation}

The last ingredient to finally derive the equation of motion in the lagrangian
formalism is the \emph{Hamilton's principle} \cite{Goldstein11_Ham_principle}:
\begin{displayquote}
  \emph{The motion of the system it's a stationary point of the action
  functional \eqref{eq:general_action}}, i.e.
  \begin{equation*}
    \frac{\delta S[q^{\mu}]}{\delta q^{\mu}} = 0
  \end{equation*}
\end{displayquote}
Keeping ``fixed-end'' (the initial and final configurations of the system are
the ones we choose hence are fixed) during the variations and using the results
from Appendix \ref{appendix: calculus of variation} we arrive at the
\emph{Euler-Lagrange equation}, i.e. the equation of motion in lagrangian
mechanics.

\begin{equation} \label{eq:euler-lagrange}
    \frac{\delta S[q^{\mu}]}{\delta q^{\mu}} =
    \frac{\partial L}{\partial q^{\mu}} -
    \frac{d}{dt}\frac{\partial L}{\partial \dot{q}^{\mu}} +
    \frac{d^2}{dt^2}\frac{\partial L}{\partial \ddot{q}^{\mu}} -
    \ldots = 0
\end{equation}

Usually the underling Lagrangian of a classical mechanical system is $L(q^{\mu},
\dot{q}^{\mu})$ so the equation \eqref{eq:euler-lagrange} it's truncated after
the second term. In this restriction of the lagrangian formalism, lagrangian
mechanics and newtonian mechanics are equivalent. $L$ can be written as
difference of two quantities of the newtonian mechanics: kinetic energy $T$ and
potential energy $V$. \\

In the classical mechanics problems (such as particle constrains on a surface or
a double pendulum) the number of degrees of freedom of the system is finite.
Lagrangian formalism can be extends to systems with an infinite countable number
of degrees of freedom (think to the classical gas). But the formalism could be
push further more and be able to describe \emph{classical fields}.

\begin{definition}\label{def: classical field}
  \cite{Aldrovandi19_classical_field}
  A \emph{classical field} is a system with a continuous infinity of degrees of
  freedom.
\end{definition}

Going from discrete to continuous the notation to indicate the generalized
coordinates has to be review

\begin{align*}
  q(t) \rightarrow \phi, \quad \mu \rightarrow \bm{x}
  \qquad \Rightarrow \qquad
  q^{\mu}(t) \rightarrow \phi(\bm{x},t)
\end{align*}

The previous generalized coordinates $q$ were just $\mu$ real number evolving in
time according to \eqref{eq:euler-lagrange}. Now $\phi$ are the generalized
coordinates but unlike $q$ they are function of space $\bm{x}$ and time $t$. The
evolution of $\phi$ depends on time and space so the dot notation for the
derivation should be replace by a one that take also space derivatives
into account. Special relativity provide a nice framework to join the space and
time into \emph{spacetime} that allows us to treat $\bm{x}$ and $t$ on equal
footing
\footnote{
  Following the standard notation quantities derived from newtonian space are
  written in bold (e.g. $\bm{x}$, $\bm{p}$) while the corresponding quantities
  in Minkowski spacetime are not (e.g. $x$, $p$).
}.

\begin{equation*}
  x^0 := ct \qquad
  x := (x^0, \bm{x}) \qquad
  \partial_{\mu} := \left(
    \frac{\partial}{\partial x^0},
    \frac{\partial}{\partial x^1},
    \frac{\partial}{\partial x^2},
    \frac{\partial}{\partial x^3}
  \right)
\end{equation*}

The lagrangian can also express the interaction between two or more fields.
Given a set of fields $\bm{\phi}(x):=\{\phi_i(x)\}$ with $i=1,\ldots,n$ the
lagrangian will have the form

\begin{equation} \label{eq:general_lagrangian_density}
  \Lagr(x) = \Lagr[
      \bm{\phi}(x),
      \partial_{\mu} \bm{\phi}(x),
      \partial_{\mu} \partial_{\nu} \bm{\phi}(x),
      \ldots
    ]
\end{equation}

Comparing \eqref{eq:general_lagrangian} and
\eqref{eq:general_lagrangian_density} it's clear that they are two different
mathematical object. The input to the lagrangian
\eqref{eq:general_lagrangian_density} are the functions $\bm{\phi}(x)$ so
$\Lagr(x)$ is not a function but a functional. The proper name of this object
is \emph{Lagrangian density}
\footnote{The origin of the term \emph{density} came from differential geometry.
  \emph{1-weight tensor densities} are object that are transformed almost like
  tensor but producing and extra multiplicative factor: the inverse determinant
  of the transformation. Later when evaluating the action
  \eqref{eq:general_action_fields}, under a change of coordinates, this term is
  simplify with the determinant produced by $dx$.
}.
In order to obtain the equations that describe the spacetime evolution of the
fields, the so called \emph{fields equations}, we apply Hamilton's principle to
\eqref{eq:general_lagrangian_density}

\begin{equation} \label{eq:general_action_fields}
  S[\bm{\phi}] := \int d^4x \Lagr(x)
\end{equation}

Observe that integration run over $x$, so $S[\bm{\phi}]$ depends only on the
functional form of $\phi$. Using again the concept of functional derivative
(appendix \ref{appendix: calculus of variation}) we get
\eqref{eq:euler-lagrange-field}. Formally the steps are equal to the derivation
of \eqref{eq:euler-lagrange}, but instead to take the derivative with respect to
time, the $\partial_{\mu}$ has been used.

\begin{equation} \label{eq:euler-lagrange-field}
  \frac{\delta S[\bm{\phi}]}{\delta \bm{\phi}} =
    \frac{\partial \Lagr}{\partial \bm{\phi}} -
    \partial_{\mu} \frac{\partial \Lagr}{\partial \partial_{\mu}\bm{\phi}} +
    \partial_{\mu}\partial_{\nu}\frac{\partial L}
    {\partial \partial_{\mu}\partial_{\nu}\bm{\phi}} -
    \ldots = 0
\end{equation}

One of the important aspect that link Lagrangian (density) and motion (field)
equation are the \emph{symmetries}: every symmetries of the Lagrangian is also a
symmetry of the equation of motion \cite{Aldrovandi19_symmetry}. Nature
suggested Newton directly the equations of motion of classical object, but
formulation of more fundamental theory stop at what humans can experience. On
the other hand in the Lagrangian formalism one try an educated guess on the
lagrangian based on the symmetries observed and the derivative the equation of
motion that will be tested by experiments.


\subsection{Hamiltonian formalism}

% TODO: add hamiltonian %
