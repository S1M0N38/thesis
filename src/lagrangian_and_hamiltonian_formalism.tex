Before dive straight in the study of Higher Derivative (HD) systems, i.e. systems
characterized by lagrangian contains terms such as $\ddot{q}$, ${q}^{(3)}$, ..., it's
better to review some basic concepts of the \emph{lagrangian} and \emph{hamiltonian
formalism} in order to fix ideas and notation.

\subsection{Lagrangian formalism}
When talking about \emph{lagrangian formalism} we mean the whole mathematical formalism
that it's been develop around the \emph{lagrangian mechanics} and then extends way
beyond to the realm of classical mechanics. The lagrangian mechanics was born as an
alternative formulation of the \emph{newtonian mechanics} describing a system
analytically instead of the vector based approach used by Newton. The keys concepts of
lagrangian mechanics are: \emph{generalized coordinates}, \emph{degrees of freedom} and
\emph{constraints}.
\begin{definition} \cite{Ginsberg08}
    \emph{Generalized coordinates} $q^{\mu}$ are a set of coordinates (i.e. a
    collection of real number) that describe uniquely the configuration of a system.
\end{definition}
\begin{definition} \cite{Ginsberg08}
    The minimum number $n$ of generalized coordinates required to specify the position
    of a system is the \emph{number of degrees of freedom} of that system.
\end{definition}
\begin{definition} \cite{Goldstein11_hol_constraints}
    Holonomic constraints are relation between generalized coordinates which can be
    expressed as:
    \begin{equation*}
        f(q^{\mu}) = 0 \qquad \mu = 1, \ldots, n < \infty
    \end{equation*}
\end{definition}
Once generalized coordinates has been chosen their evolution describe the evolution
of the system. According to classical mechanics $q^{\mu}$ evolve leisurely in a smooth
fashion following the \emph{equation of motion}. This ``smoothness'' allows following
definition:
\begin{definition}
    The \emph{generalized velocities} $\dot{q}^{\mu}$ are the time derivatives of the
    generalized coordinates $q^{\mu}$, i.e.
    \begin{equation*}
        \dot{q} := \frac{dq(t)}{dt}
    \end{equation*}
    The same idea can used to define
    \begin{equation*}
        \ddot{q} := \frac{d^2q(t)}{dt^2} \qquad \ldots \qquad
        q^{(n)} := \frac{d^nq(t)}{dt^n}
    \end{equation*}
\end{definition}

In newtonian mechanics the equations of motion are ordinary second order differential
equation related the position vector $\bm{x}$ and the forces $\bm{F}$:
\begin{equation*}
    m \frac{d^2\bm{x}}{dt^2} = \bm{F}
\end{equation*}

Lagrangian mechanics do not start from equations of motion for the description of the
system; instead it's defined a function called the \emph{Lagrangian}:
\begin{equation} \label{eq:general_lagrangian}
    L(q^{\mu}, \dot{q}^{\mu}, \ddot{q}^{\mu}, \ldots, q^{(n)\mu})
\end{equation}
An \emph{action functional} $S[q^{\mu}]$ can be associated to a system described by the
lagrangian $L$ evolving from a position at time $t_1$ to another position at time $t_2$
\begin{equation} \label{eq:general_action}
    S[q^{\mu}] := \int_{t_1}^{t_2} dt L(q^{\mu}, \dot{q}^{\mu}, \ddot{q}^{\mu}, \ldots, q^{(n)\mu})
\end{equation}
The last ingredient to finally derive the equation of motion in the lagrangian formalism
is the \emph{Hamilton's principle} \cite{Goldstein11_Ham_principle}:
\begin{displayquote}
    \emph{The motion of the system it's a stationary point of the action functional
    \eqref{eq:general_action}}, i.e
    \begin{equation*}
        \frac{\delta S[q^{\mu}]}{\delta q^{\mu}} = 0
    \end{equation*}
\end{displayquote}
Keeping ``fixed-end'' (the initial and final configurations of the system are the ones
we choose hence are fixed) during the variations and using the results from Appendix
\ref{appendix: calculus of variation} we arrive at the \emph{Euler-Lagrange equation},
i.e. the equation of motion in lagrangian mechanics.
\begin{equation} \label{eq:euler-lagrange}
    \frac{\delta L}{\delta q^{\mu}} =
    \frac{\partial L}{\partial q^{\mu}} -
    \frac{d}{dt}\frac{\partial L}{\partial \dot{q}^{\mu}} +
    \frac{d^2}{dt^2}\frac{\partial L}{\partial \ddot{q}^{\mu}} -
    \ldots = 0
\end{equation}
Usually the underling Lagrangian of a classical mechanical system is $L(q^{\mu},
\dot{q}^{\mu})$ so the equation \eqref{eq:euler-lagrange} it's truncated after the
second term. In this restriction of the lagrangian formalism, lagrangian mechanics and
newtonian mechanics are equivalent. $L$ can be written as difference of two quantity typical of
the newtonian mechanics: kinetic energy $T$ and potential energy $V$.
