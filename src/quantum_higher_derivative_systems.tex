In this section the quantum counterpart of the previous classical system will be
studied. In appendices are presented two way to go from the classical
lagrangian and hamiltonian formalism to the quantum mechanical ones:
\emph{Feynman path integrals} (\ref{appendix:path_integrals}) and
\emph{canonical quantization} (\ref{appendix:canonical_quantization}). These
procedures will be used to derive the Quantum Pais-Uhlenbeck (QPU) oscillator
starting from \eqref{eq:lagrangian_PU}. The quantum version of the Ostrogradsky
theorem will be proved and lastly more example of ghost-ridden theories are
given.

\subsection{Ghosts}
As said previously in section
\ref{subsection:linear_ostrogradskian_instability}, in quantum mechanics
\emph{linear Ostrogradskian instability} goes under the name ``\emph{ghost}''.
Just like classical HD theories, quantum HD theories have the spectrum of the
hamiltonian unbounded. Theorem \ref{th:ostrogradsky_classical} have its quantum
counterpart:

\begin{theorem} \label{th:ostrogradsky_quantum}
  \cite{Smilga17} The quantum counterpart of Hamiltonian
  \eqref{eq:general_hamiltonian_1dim} for a non-degenerate higher-derivative
  system has no ground state.
\end{theorem}
\begin{proof}
  The proof is given for a one-dimentional system ($\mu=1$) describe by a second
  order lagrangian ($n=2$); the generalizaton for a multi-dimentional HD system
  is straightforward. The result hamiltonian is
  \begin{equation*}
    H = h(Q_1, Q_2, P_2) P_2 + Q_2 P_1 - L(Q_1, Q_2, h)
  \end{equation*}
  Using the canonical quantization the function H is converted to a quantum
  observable $\hat{H}$ which eigenvalue are the allowed energies of the system.
  \begin{align*}
    & \begin{cases}
      Q_1 \xrightarrow{\wedge} \hat{Q_1} \equiv \hat{x}
      \quad &\mid \quad \hat{x}\psi(x, v) = x \psi(x, v) \\
      P_1 \xrightarrow{\wedge} \hat{P_1} \equiv \hat{p}_x
      \quad &\mid \quad \hat{p}_x\psi(x, v) = -i\partial_x \psi(x, v) \\
    \end{cases} \\
    & \begin{cases}
      Q_2 \xrightarrow{\wedge} \hat{Q_2} \equiv \hat{v}
      \quad &\mid \quad \hat{v}\psi(x, v) = v \psi(x, v) \\
      P_2 \xrightarrow{\wedge} \hat{P_2} \equiv \hat{p}_v
      \quad &\mid \quad \hat{p}_v\psi(x, v) = -i\partial_v \psi(x, v) \\
    \end{cases}
  \end{align*}
  \begin{equation} \label{eq:ham_operator_2nd_order}
    \hat{H} = h(\hat{x}, \hat{v}, \hat{p}_v) \hat{p}_v + \hat{v} \hat{p}_x -
    L (\hat{x}, \hat{v}, h)
  \end{equation}

  Moreover suppose that the spectrum of the hamiltonian is discrete. If was
  continuous, normalize it in a box so the range of $x$ and $v$ is finite which
  make the spectrum discrete. At the end remove regularization sending the box
  size to infinity.
  Let $\psi_0(x, v)$ be the normalized ground state of the system with the
  associated eigenvalue $E_0$, i.e. $\hat{H}\psi_0(x, v) = E_0 \psi_0(x, v)$.
  To seek for an energy value close to $E_0$ we use the \emph{variational
  method}
  \footnote{
    Variational methods must not be confused with variational calculus, they
    are totally different things.
  } with the following variational ansatz:

  \begin{equation}
    \psi_c(x, v) = e^{icx}\psi_0(x,v)
  \end{equation}

  Observe that $\psi_c(x,v)$ is not an eigenfunction of the Hamiltonian
  \eqref{eq:ham_operator_2nd_order} and neither linear combination $\psi_f(x,v)$
  of $\psi_c(x,v)$ are.

  \begin{align*}
    &\hat{H} \psi_c(x,v) = E_0\psi_c(x, v) + cv \psi_c(x,v)
    \neq \lambda \psi_c(x,v) \\ \\
    &\psi_f = \int f(c) \psi_c \, dc = \psi_0 \int f(c) e^{icx} \, dc =
    \psi_0 \tilde{f}(x) \\
    &\hat{H} \psi_f(x,v) = E_0\psi_f(x, v) -
    iv \frac{\partial \tilde{f}(x)}{\partial x} \psi_c(x,v)
    \neq \lambda \psi_c(x,v)
  \end{align*}

  So $\psi_c$ and $\psi_f$ does not vanish of exited state and variational
  energy must not exceed $E_0$ and can be obtain by direct calculation.

  \begin{equation*}
    E_0 \leq E_c = \int \psi_c^* \hat{H} \psi_c \, dx \, dv =
    E_0 + c \int v \left|\psi_0 \right| ^2 \, dx \, dv
  \end{equation*}

  If the integral $\int v \left|\psi_0 \right| ^2 \, dx \, dv$ does not vanish
  arbitrarily negative energies can be reach by an appropriate choice of $c$.
  Otherwise if the integral vanish, $E_c$ does not depend on $c$ and does not
  grow, as it should if $\psi_0$ would be a ground state \cite{Smilga17}. Thus,
  $\psi_0$ is not a ground state as it was postulated. \\
\end{proof}
