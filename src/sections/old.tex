  \begin{comment}
  \section{Euler–Lagrange equation}

  Euler–Lagrange equation is the most important equation of Lagrangian
  mechanics; it's second-order PDE involving position $x^{\mu}$ and velociy
  $\dot{x^{\mu}}$. With this notation, $x^{\mu}(t)$ can be tought as a
  parametrization (by a real number $t$) of a regular curve on a smooth
  manifold $M$ and $\dot{x^{\mu}}$ as the componet of the tangent vector (at
  p $\in M$) in the choosen chart.

  \begin{equation} \label{eq:Euler-Lagrange eq}
    \frac{\partial\Lagr}{\partial x^{\mu}} -
    \frac{d}{dt}\left(\frac{\partial\Lagr}{\partial \dot{x^{\mu}}}\right) = 0
  \end{equation}

  Historically \eqref{eq:Euler-Lagrange eq} can came up with the study of
  problems about variation\footnote{in particoular thetautochrone curve} but
  an eqivalent derivation can be done starting from Newtown motion equations.
  Eventhough the two approces led to same result, the one
  propose originally by Lagrange it's easy to generalize:\footnote{The
  stationar action principle is used broadly used in other braches of physics:
  Quantum Mechanics (Feymann path integral) and Optics (Fermat principle) to
  name a few} just include higher order terms in the Lagrangian $\Lagr$
  appearing in the action funtional.

  \begin{equation*}
    A[f] = \int_{a}^{b} \Lagr(t, f, f', \ldots, f^{(n)}) dt
  \end{equation*}

  As usual, we start considering a curve $f: \R \rightarrow M: t \mapsto f(t)$
  (this time we call it $f$ instead of use the chart $x^{\mu}(t)$ just for
  a light notation during calculation) and we produce a deformation $\tilde{f}$
  while keeping the start and end points fixed in place for all value of
  $\epsilon$.

  \begin{gather*}
    f:  \R \rightarrow M: t \mapsto f(t) \qquad
    \tilde{f} : \R \rightarrow M: t \mapsto f(t) + \epsilon\eta(t) \\ \\
    f(a) = \tilde{f}(a) \implies \eta(a) = \eta'(a) = \ldots = \eta^{(n)}(a) = 0 \\
    f(b) = \tilde{f}(b) \implies \eta(b) = \eta'(b) = \ldots = \eta^{(n)}(b) = 0
  \end{gather*}

  Now suppose that $f$ is the exstremal curve, plugging its deformation
  $\tilde{f}$ inside the $A$, the value of this functional rise for every
  $\eta$ and for every value of $\epsilon$ (even infinitesimal ones), so the
  whole problem can be solved using basic single-variable calculus.
  The sationary point of the functional $A[\tilde{f}]: \R \rightarrow \R$
  can be found setting
  $\left. \frac{dA[\tilde{f}]}{d\epsilon} \right \rvert_{\epsilon=0}$

  \begin{align}
    % Actoin on deformed curve
    A[\tilde{f}] &=
    \int_{a}^{b} \Lagr(t, f(t) + \epsilon\eta(t), f'(t) + \epsilon\eta'(t),
    \ldots, f^{(n)}(t) + \epsilon\eta^{(n)}(t)) dt \\
    % set first derivate = 0
    \frac{dA[\tilde{f}]}{d\epsilon} &= \int_{a}^{b} \left[
    \frac{\partial\Lagr}{\partial\tilde{f}}\eta(t) +
    \frac{\partial\Lagr}{\partial\tilde{f}'}\eta'(t) + \ldots
    \frac{\partial\Lagr}{\partial\tilde{f}^{(n)}}\eta^{(n)}(t) \right] dt = 0
    \label{eq:first-derviate-wrt-epslion}
  \end{align}

  Appling Leibniz rule for integration recurlsivly and noting that at boundary
  (i.e. at point $a$ and $b$) these integrals vanished we get this formula for
  the $k^{th}$ term in \eqref{eq:first-derviate-wrt-epslion}.

  \begin{equation} \label{eq:kth-term Liebniz rule}
    \int_{a}^{b} \frac{\partial\Lagr}{\partial\tilde{f}^{(k)}}\eta^{(k)}(t) dt
    = (-1)^{(k)}\int_{a}^{b} \frac{d^k}{dt^k}\left(
    \frac{\partial\Lagr}{\partial\tilde{f}^{(k)}}\right)\eta(t) dt
  \end{equation}

  Writing \eqref{eq:first-derviate-wrt-epslion} using \eqref{eq:kth-term Liebniz
  rule}, $\eta(t)$ can be collected from each term and evaluating at $\epsilon=0$
  follows the \emph{generalized Euler-Lagrange equation}:

  \begin{equation} \label{eq:Euler-Lagrange eq generalized}
      \frac{\partial\Lagr}{\partial f}
    - \frac{d}{dt}\left(\frac{\partial\Lagr}{\partial f'}\right)
    + \frac{d^2}{dt^2}\left(\frac{\partial\Lagr}{\partial f''}\right)
    + \ldots
    + (-1)^n \frac{d^n}{dt^n}\left(\frac{\partial\Lagr}{\partial f^{(n)}}\right)
    = 0
  \end{equation}

\end{comment}
