\documentclass[a4paper]{article}

\usepackage{amsmath} % math stuff
\usepackage{amsthm}
\usepackage{amssymb} % math symbols (like Real, Complex, ...)
\usepackage{bm}
\usepackage{mathabx}

\usepackage{tikz} % images

\usepackage{lipsum} % Lorem ipsum dolor sit ...
\usepackage{comment}
\usepackage{csquotes}
\usepackage{enumitem}

\usepackage[toc,page]{appendix}
\usepackage{subfiles}
\usepackage[hidelinks]{hyperref}
\usepackage{biblatex}

\addbibresource{references.bib}

\theoremstyle{definition}
\newtheorem{definition}{Definition}[section]
\newtheorem{theorem}{Theorem}[section]
\numberwithin{equation}{section}


\title{Amazing project}
\author{Simone Bertolotto}


% Shortcut
\DeclareMathOperator{\Lagr}{\mathcal{L}}  % Lagrangian
\DeclareMathOperator{\R}{\mathbb{R}}      % Real Number
\DeclareMathOperator{\N}{\mathbb{N}}      % Natural Number
\DeclareMathOperator{\C}{\mathbb{C}}      % Complex Number

% Layout
\setlength{\parindent}{0pt}


\begin{document}

  \maketitle

  \begin{abstract}
    \lipsum[1]
  \end{abstract}

  \newpage

  \section{Introduction}
  \label{section: introduction}
  \subfile{introduction}

  \section{Lagrangian and Hamiltonian formalism}
  \label{section: lagrangin and hamiltonian formalism}
  \subfile{lagrangian_and_hamiltonian_formalism}

  \section{Classical higher derivative systems}
  \label{section: classical higher derivative systems}
  \subfile{classical_higher_derivative_systems}

  \section{Quantum higher derivative systems}
  \label{section: quantum higher derivative systems}
  \subfile{quantum_higher_derivative_systems}

  \newpage
  \begin{appendices}
    \section{Calculus of variations}
    \label{appendix: calculus of variation}
    \subfile{calculus_of_variations}

    \section{Path Integrals}
    \label{appendix:path_integrals}
    \subfile{path_integrals}

    \section{Canonical quantization}
    \label{appendix:canonical_quantization}
    \subfile{canonical_quantization}
  \end{appendices}

  \newpage

  \printbibliography

\end{document}
