\documentclass[a4paper]{article}

% -----------------------------------Packages--------------------------------- %

% Allow to use é, é, ç, ... directly from keyboard
\usepackage[utf8]{inputenc}

% T1 font encoding is an 8-bit encoding and uses fonts that have 256 glyphs
% (e.g. 'ö' is a single glyph and is not made by adding accent to 'o')
\usepackage[T1]{fontenc}

% Set the language of the document (e.g. title, section, abstract, ...)
\usepackage[english]{babel}

% math stuff: equation, declaring operators, ...
\usepackage{amsmath}

% math additional symbols: mathbb (blackboard font), mathfrak (Fraktur letters)
% include asmfont as a dependencies
\usepackage{amssymb}

% additional features on theorms such as theoremstyle
\usepackage{amsthm}

% bold font with \bm command
\usepackage{bm}

% enumerate lists
\usepackage{enumitem}

% import subfiles .tex
\usepackage{subfiles}

% make \ref clickale
\usepackage[hidelinks]{hyperref}

% insert appendix
\usepackage[toc,page]{appendix}

% biblatex dependency
\usepackage{csquotes}

% References loader with \addbibresource{*.bib}
\usepackage{biblatex}


% ------------------------------Bibliography---------------------------------- %

\addbibresource{references.bib}


% ------------------------------Theorems config------------------------------- %

\theoremstyle{definition}
\newtheorem{definition}{Definition}[section]
\newtheorem{theorem}{Theorem}[section]
\numberwithin{equation}{section}


% ------------------------------Customs commands------------------------------ %

% Custom command
\newcommand{\field}[1]{\mathbb{#1}}       % Complex, Real, Natural, ...
\DeclareMathOperator{\Lagr}{\mathcal{L}}  % Lagrangian
\DeclareMathOperator{\U}{\mathcal{U}}     % Time evolution operator


% ----------------------------------Layout------------------------------------ %

\setlength{\parindent}{0pt}


% ----------------------------------Document---------------------------------- %

\title{TODO}
\author{Simone Bertlotto}


\begin{document}

  \maketitle

  \begin{abstract}
    TODO
  \end{abstract}

  \newpage

  \section{Introduction}\label{section: introduction}
  \subfile{introduction}

  \section{Lagrangian and Hamiltonian formalism}\label{section: lagrangin and
  hamiltonian formalism}
  \subfile{lagrangian-and-hamiltonian-formalism}

  \section{Classical higher derivative systems}\label{section: classical higher
  derivative systems}
  \subfile{classical-higher-derivative-systems}

  \section{Quantum higher derivative systems}\label{section: quantum higher
  derivative systems}
  \subfile{quantum-higher-derivative-systems}

  \section{Summary}\label{section: summary}
  \subfile{summary}

  \newpage

  \begin{appendices}
    \section{Calculus of variations}\label{appendix: calculus of variation}
    \subfile{calculus-of-variations}

    \section{Canonical quantization}\label{appendix:canonical_quantization}
    \subfile{canonical-quantization}
  \end{appendices}

  \newpage

  \printbibliography{}

\end{document}
