A quantum system could be derived from the classical counterpart starting by
hamiltonian formalism; this procedure is base on a map $\wedge$ that take a
function $f$ define on classical phase space into a quantum \emph{observable}
$\hat{f}$ (i.e. a hermitian operator acting of elements of a Hilbert space). A
generic $f$ is define in terms of $Q$ and $P$ while $\hat{f}$ is given in terms
of $\hat{Q}$ and $ \hat{P}$.
Various sets of instructions for perform canonical quantization exist (e.g.
deformation quantization, geometric quantization, ...) but they all originate
from the one proposed by Dirac \cite{Dirac25}. The map proposed should have
these following proprieties:
\begin{enumerate}[label=(\roman*)]
  \item $\wedge$ takes canonical variable $Q_i$ and $P_i$ in operators:
    \begin{equation*}
      \begin{cases}
        Q \xrightarrow{\wedge} \hat{Q} \equiv \hat{x}
        \quad &\mid \quad \hat{x} \psi(x) := x \psi(x) \\
        P \xrightarrow{\wedge} \hat{P} \equiv \hat{p}
        \quad &\mid \quad \hat{p} \psi(x) := -i \partial_x \psi(x) \\
      \end{cases}
    \end{equation*}
  \item $\wedge$ is linear
  \item $\wedge$ provide a ``compatibility relation between Poisson algebra and
    the algebra of quantum observable'' i.e.
    \begin{equation*}
      \left\{ f, \, g \right\} \xrightarrow{\wedge}
      -i \left[ \hat{f}, \, \hat{g} \right]
    \end{equation*}
\end{enumerate}
Unfortunately conditions (i), (ii), (iii) have been proved mutually inconsistent
hence the maps result ill-define
\footnote{
  Groenewold's theorem say that no such map $\wedge$ can produce an exact
  correspondence between Poisson bracket and commutator.
}.
However if we restrict to lower degrees polynomials we have the following

\begin{theorem} \label{th:2nd_deegree_poli_canonical_quantization}
  \cite{Hall13} If $f(Q,P)$ is a polynomial of degree at most 2 and $g(Q,P)$ is
  an arbitrary polynomial $\wedge$ is well-define.
\end{theorem}

Starting from classical system and then quantized it have some advantages and
some drawbacks. The main drawback is that with this approach can describe the
purely quantum phenomena such the \emph{spin}. The main advantage is that we can
develop our quantum intuition by referring to the classical system from which we
started. Analogies between quantum systems and classical systems are also
reflected in the equations that describe them. For example the evolution of an
observable (without an explicit dependence on time)is given by Ehrenfest theorem
that written in Heisenberg picture is

\begin{equation*}
  \frac{d}{dt} \hat{f} = -i \left[ \hat{f}, \, \hat{H} \right]
\end{equation*}

which exhibits a close resemblance with \eqref{eq:poisson_bracket_evolution}
using property (iii). Another advantage concerns the coordinates transformation.
Sometime equations of a quantum system take simpler form under coordinates
transformation. One can use the theory of the canonical transformation develop
in classical mechanics and then canonically quantize the result.
%TODO add caveat about ordering in the operatator
